\documentclass[a4paper,10pt]{article}
\usepackage{fontspec}
\usepackage{polyglossia}
\setdefaultlanguage{spanish}
\defaultfontfeatures{Ligatures={NoCommon}}
\setmainfont{TeX Gyre Pagella}[
 SmallCapsFeatures={LetterSpace=4.0},
 Numbers={OldStyle,Proportional}
 ]
\setmonofont{TeX Gyre Heros}
\usepackage{listings}
\lstset{
    basicstyle=\small\ttfamily,
    columns=flexible,
    breaklines=true
}
\author{CervanTeX}
\title{Colección de mensajes de la lista de correo de CervanTeX}
\date{febrero de 2017\\ \footnotesize{(mensajes en hora local de España)}}
\usepackage{nameref}
\setcounter{secnumdepth}{0}
\begin{document}
\maketitle

\tableofcontents

\section{Promoción de CervanTeX en TeXStackexchange}

\subsection{Día 08 12:41:10, Ignasi}

\begin{lstlisting}
Supongo que ya conocéis el sitio <a href="http://tex.stackexchange.com" target="_blank">http://tex.stackexchange.com</a> de 
preguntas y respuestas sobre LaTeX.
En este sitio cada nuevo año  reinician una campaña para promocionar 
otros sitios/listas/programas/... relacionados con lo que los usuarios 
consideran oportuno. Preferentemente relacionado con LaTeX.
Desde hace varios años he propuesto un anuncio de CervanTeX. Este año 
también, pero de momento sólo tiene 4 votos y necesita 6.
Si alguno de vosotros es miembro de TX.SX y considera oportuno que el 
logo de Cervantex aparezca de vez en cuando en dicho sitio, podéis votar 
en:
  <a href="http://meta.tex.stackexchange.com/questions/7182/community-promotion-ads-2017?cb=1" target="_blank">http://meta.tex.stackexchange.com/questions/7182/community-promotion-ads-2017?cb=1</a>
Saludos,
Ignasi
Aquest correu electrònic s'ha verificat mitjançant l'Avast antivirus.
<a href="https://www.avast.com/antivirus" target="_blank">https://www.avast.com/antivirus</a>

\end{lstlisting}

\subsection{Día 08 13:18:40, JL Diaz}

\begin{lstlisting}
Hombre, Ignasi, no sabía que estabas por aquí también, un saludo.
Ya he votado
Saludos,
--JL Diaz
El 8 de febrero de 2017, 12:41, Ignasi <<a href="/cgi-bin/wa?LOGON=A3%3Dind1702%26L%3DES-TEX%26E%3Dquoted-printable%26P%3D14873%26B%3D--001a113f86d0abd294054803debe%26T%3Dtext%252Fplain%3B%2520charset%3Dutf-8%26header%3D1" target="_parent" >[conectar para ver]</a>> escribió:
> Supongo que ya conocéis el sitio http://tex.stackexchange.com de
> preguntas y respuestas sobre LaTeX.
>
> En este sitio cada nuevo año  reinician una campaña para promocionar otros
> ...(texto omitido)...

\end{lstlisting}

\subsection{Día 08 16:07:29, Josep Ysern}

\begin{lstlisting}
Yo también he votado... pero ha salido un aviso diciendo que, como tengo 
muy "baja reputación" (sic!)... no sé qué... (¿tiran mi voto a la 
papelera?).
Bueno, al menos lo he intentado...
Un slaudo,
Josep
El 08/02/17 a les 12:41, Ignasi ha escrit:
> Supongo que ya conocéis el sitio http://tex.stackexchange.com de 
> preguntas y respuestas sobre LaTeX.
>
> En este sitio cada nuevo año  reinician una campaña para promocionar 
> ...(texto omitido)...

\end{lstlisting}

\subsection{Día 08 16:22:53, alberto alejandro moyano}

\begin{lstlisting}
Ya está mi voto.
Saludos
A.
El 8 de febrero de 2017, 12:07, Josep Ysern <
<a href="/cgi-bin/wa?LOGON=A3%3Dind1702%26L%3DES-TEX%26E%3Dquoted-printable%26P%3D37961%26B%3D--001a113d22383ec578054806700d%26T%3Dtext%252Fplain%3B%2520charset%3Dutf-8%26header%3D1" target="_parent" >[conectar para ver]</a>> escribió:
> Yo también he votado... pero ha salido un aviso diciendo que, como tengo
> muy "baja reputación" (sic!)... no sé qué... (¿tiran mi voto a la
> papelera?).
>
> ...(texto omitido)...

\end{lstlisting}

\subsection{Día 08 16:39:19, Aradenatorix Veckhôm Awecaelus}

\begin{lstlisting}
Gracias Ignasi, ya he votado, parece que tenemos 7 votos a favor hasta ahora ;)
Saludos

\end{lstlisting}

\subsection{Día 08 18:41:18, Vicent Giner Bosch}

\begin{lstlisting}
Yo también he votado.  :)
11 votos de momento.
vicent

\end{lstlisting}

\subsection{Día 08 19:19:31, Pedro Alberto Enriquez Palma}

\begin{lstlisting}
Otro más.    :) 
 
-----Mensaje original----- 
De: Usuarios hispanohablantes de TeX [mailto:[conectar para ver]] En nombre de Ignasi 
Enviado el: miércoles, 08 de febrero de 2017 12:41 
Para: <a href="/cgi-bin/wa?LOGON=A2%3Dind1702%26L%3DES-TEX%26F%3D%26S%3D%26P%3D4241" target="_parent" >[conectar para ver]</a> 
Asunto: [TEX] Promoción de CervanTeX en TeXStackexchange 
 
Supongo que ya conocéis el sitio <a href="http://tex.stackexchange.com" target="_blank">http://tex.stackexchange.com</a> de preguntas y respuestas sobre LaTeX. 
 
En este sitio cada nuevo año  reinician una campaña para promocionar otros sitios/listas/programas/... relacionados con lo que los usuarios consideran oportuno. Preferentemente relacionado con LaTeX. 
 
Desde hace varios años he propuesto un anuncio de CervanTeX. Este año también, pero de momento sólo tiene 4 votos y necesita 6. 
 
Si alguno de vosotros es miembro de TX.SX y considera oportuno que el logo de Cervantex aparezca de vez en cuando en dicho sitio, podéis votar 
en: 
 
  <a href="http://meta.tex.stackexchange.com/questions/7182/community-promotion-ads-2017?cb=1" target="_blank">http://meta.tex.stackexchange.com/questions/7182/community-promotion-ads-2017?cb=1</a> 
 
Saludos, 
 
Ignasi 
 
 
--- 
Aquest correu electrònic s'ha verificat mitjançant l'Avast antivirus. 
<a href="https://www.avast.com/antivirus" target="_blank">https://www.avast.com/antivirus</a> 
 
---------------------------------------------------- 
---------------------------------------------------- 
 
---------------------------------------------------- 
---------------------------------------------------- 
 
 

\end{lstlisting}

\subsection{Día 08 19:30:44, Enrique Lazcorreta Puigmartí}

\begin{lstlisting}
Hola.
Acabo de crear mi cuenta y ahora me piden una reputación mínima para votar,
como a Josep. Con lo difícil que es hoy en día conseguir buena
reputación... Espero que sean suficientes esos 11 votos que ya tenemos.
Suerte
El mié., 8 feb. 2017 a las 19:19, Pedro Alberto Enriquez Palma (<
<a href="/cgi-bin/wa?LOGON=A3%3Dind1702%26L%3DES-TEX%26E%3Dquoted-printable%26P%3D76629%26B%3D--001a114c8f4eadb8210548091078%26T%3Dtext%252Fplain%3B%2520charset%3Dutf-8%26header%3D1" target="_parent" >[conectar para ver]</a>>) escribió:
> Otro más.    :)
>
> -----Mensaje original-----
> De: Usuarios hispanohablantes de TeX [mailto:[conectar para ver]]
> ...(texto omitido)...

\end{lstlisting}

\subsection{Día 08 19:40:10, Aradenatorix Veckhôm Awecaelus}

\begin{lstlisting}
Es cuestión de tiempo y participar, yo la verdad es que con tanto
trabajo ya no puedo hacerlo como quisiera... además de que tengo unas
dudas en el tintero que no pude resolver ahí y no sé si convenga
traerlas a esta lista, por aquello del top posting y esas cuestiones
de la netiquette.
Saludos

\end{lstlisting}

\subsection{Día 08 22:18:15, Ignasi}

\begin{lstlisting}
Veo que somos muchos los que estamos presentes en los dos foros.
Gracias a todos, con 6 votos era suficiente y los hemos superado con creces.
Ignasi
El 08/02/2017 a les 19:30, Enrique Lazcorreta Puigmartí ha escrit:
> Hola.
>
> Acabo de crear mi cuenta y ahora me piden una reputación mínima para 
> votar, como a Josep. Con lo difícil que es hoy en día conseguir buena 
> ...(texto omitido)...
Aquest correu electrònic s'ha verificat mitjançant l'Avast antivirus.
https://www.avast.com/antivirus

\end{lstlisting}
\section{unsubscribe}

\subsection{Día 17 05:30:20, }

\begin{lstlisting}
-/-/-/-/-/-/-/-/-/-/-/-/-/-/-/-/-/-/-/-/-/-/-/-
Prof. Luis Garc�a Oropeza                     -
Universidad de Los Andes                      -
Fac. de Ciencias Econ�micas y Sociales (FACES)-
N�cleo Universitario La Liria                 -
Edif. H, Piso 3                               -
Dpto. de Econom�a                             -
M�rida, Estado M�rida                         -
Venezuela                                     -
<a href="/cgi-bin/wa?LOGON=A2%3Dind1702%26L%3DES-TEX%26F%3D%26S%3D%26P%3D7543" target="_parent" >[conectar para ver]</a>                                -
Telf. Contacto (+58) 416 9742161              -
Archivos de ES-TEX: <a href="http://listserv.rediris.es/archives/es-tex.html" target="_blank">http://listserv.rediris.es/archives/es-tex.html</a>

\end{lstlisting}

\subsection{Día 17 09:27:24, Luis Seidel Gómez de Quero}

\begin{lstlisting}
Hola Luis y todos,
Para darse de baja de la lista no hay que mandar un mensaje de
"unsubscribe" a la misma. De vez en cuando nos lo recuerda el propio
listserv al final de los mensajes:
Para darse de baja ES-TEX pincha y envia el siguiente url
mailto:[conectar para ver]
Pero no pasa nada, te doy de baja en cuanto mande este mensaje.
Saludos,
Luis Seidel
El día 17 de febrero de 2017, 5:30,  <<a href="/cgi-bin/wa?LOGON=A2%3Dind1702%26L%3DES-TEX%26F%3D%26S%3D%26P%3D8036" target="_parent" >[conectar para ver]</a>> escribió:
> -/-/-/-/-/-/-/-/-/-/-/-/-/-/-/-/-/-/-/-/-/-/-/-
> Prof. Luis García Oropeza                     -
> Universidad de Los Andes                      -
> Fac. de Ciencias Económicas y Sociales (FACES)-
> ...(texto omitido)...
Archivos de ES-TEX: <a href="http://listserv.rediris.es/archives/es-tex.html" target="_blank">http://listserv.rediris.es/archives/es-tex.html</a>

\end{lstlisting}
\section{(fuera-de-tema) un favor de prueba con mupdf}

\subsection{Día 17 19:39:19, Pablo Rodríguez}

\begin{lstlisting}
Hola,
probablemente os suene mupdf, que es una biblioteca con un visor de PDF
(así como XPS y ePub2) multiplataforma (<a href="http://mupdf.com)" target="_blank">http://mupdf.com)</a>. Es
extraordinariamente ligero, rápido y de programación libre (AGLPv3,
aunque con doble licencia). Es de la casa Artifex (los creadores de
Ghostscript. En mi opinión, en un proyecto muy prometedor.
Es el código también de SumatraPDF (<a href="https://sumatrapdfreader.org/)" target="_blank">https://sumatrapdfreader.org/)</a>, si
bien la versión de mupdf ya está desfasada (y no se puede actualizar por
desarrollos incompatibles).
Por si alguien lo usa o lo quiere probar (tanto en Windows, como en
Linux), tengo una pregunta respecto a la última versión publicada (1.10a).
Con el binario mupdf-gl (no mupdf o mupdf-x11 [ahí sí que funcionan]),
¿os funcionan las teclas 'f' para activar o desactivar la pantalla
completa o 'r' para recargar el documento?
A mí no me funcionan (tanto en Windows, como en Linux), he informado a
los desarrolladores y me dicen que les funciona. Para poder ver qué
pasa, tengo que confirmar que hay más gente a la que le funciona.
Perdonad las molestias y muchas gracias,
Pablo
-- 
<a href="http://www.ousia.tk" target="_blank">http://www.ousia.tk</a>
Archivos de ES-TEX: <a href="http://listserv.rediris.es/archives/es-tex.html" target="_blank">http://listserv.rediris.es/archives/es-tex.html</a>

\end{lstlisting}

\subsection{Día 17 19:50:37, Joaquín Ataz López}

\begin{lstlisting}
¡Vaya! ¡Qué puntería! Precisamente estaba instalando en mi recién 
comprado ordenador de sobremesa mupdf (con otras herramientas para 
ficheros pdf) ahora mismo.
A mi me funciona perfectamente la letra F. Pero yo tengo instalado el 
paquete mupdf a secas. En la paquetería de Debian no veo ningún mupdf-gl.
Hace más de un año que, como visor de PDF por defecto utilizo muPDF, 
tiene algún defectillo, como, por ejemplo, que no puedes con él hacer 
búsquedas que incluyan tildes en ficheros PDF con texto, o que no tiene 
acceso al outline... Pero las compensa con creces por su rapidez y ligereza.
El 17/02/17 a las 19:39, Pablo Rodríguez escribió:
> Hola,
>
> probablemente os suene mupdf, que es una biblioteca con un visor de PDF
> (así como XPS y ePub2) multiplataforma (<a href="http://mupdf.com)" target="_blank">http://mupdf.com)</a>. Es
> ...(texto omitido)...
Archivos de ES-TEX: <a href="http://listserv.rediris.es/archives/es-tex.html" target="_blank">http://listserv.rediris.es/archives/es-tex.html</a>

\end{lstlisting}

\subsection{Día 17 19:52:20, eric}

\begin{lstlisting}
hola, no se si te sirve, pero probe mupdf (1.9a+ds1-2) que es la version 
mas nueva que hay en debian testing (stretch) y tanto "f" como "r" 
funcionan.
Slds,
eric.
On 02/17/2017 03:39 PM, Pablo Rodríguez wrote:
> Hola,
>
> probablemente os suene mupdf, que es una biblioteca con un visor de PDF
> (así como XPS y ePub2) multiplataforma (<a href="http://mupdf.com)" target="_blank">http://mupdf.com)</a>. Es
> ...(texto omitido)...
-- 
Forest Engineer
Master in Environmental and Natural Resource Economics
Ph.D. student in Sciences of Natural Resources at La Frontera University
Member in AguaDeTemu2030, citizen movement for Temuco with green city 
standards for living
Nota: Las tildes se han omitido para asegurar compatibilidad con algunos 
lectores de correo.
Archivos de ES-TEX: <a href="http://listserv.rediris.es/archives/es-tex.html" target="_blank">http://listserv.rediris.es/archives/es-tex.html</a>

\end{lstlisting}

\subsection{Día 17 20:19:33, Pablo Rodríguez}

\begin{lstlisting}
On 02/17/2017 07:50 PM, Joaquín Ataz López wrote:
> ¡Vaya! ¡Qué puntería! Precisamente estaba instalando en mi recién 
> comprado ordenador de sobremesa mupdf (con otras herramientas para 
> ficheros pdf) ahora mismo.
> 
> ...(texto omitido)...
Joaquín,
muchas gracias por tu ayuda. Si no puedes buscar en Unicode (χαλεπὰ τὰ
καλά), estás usando mupdf-X11.
Te lo muestro, <a href="http://ousia.tk/mupdf-utf8.png" target="_blank">http://ousia.tk/mupdf-utf8.png</a> como busca una alfa
minúscula griega.
Pulsando 'i' (<a href="http://ousia.tk/mupdf-metadata.png" target="_blank">http://ousia.tk/mupdf-metadata.png</a>) muestra la información
del archivo PDF.
Pulsando 'o' muestra los marcadores (<a href="http://ousia.tk/mupdf-outlines.png)" target="_blank">http://ousia.tk/mupdf-outlines.png)</a>.
Eso en mupdf-gl, que está disponible desde la versión 1.8 (creo
recordar). Pero tiene que estar activada la compilación de ese binario
(por defecto sólo se compilaba mupdf-x11 [aunque se denomine sólo mupdf]).
Muchas gracias por la ayuda,
Pablo
-- 
<a href="http://www.ousia.tk" target="_blank">http://www.ousia.tk</a>
Archivos de ES-TEX: <a href="http://listserv.rediris.es/archives/es-tex.html" target="_blank">http://listserv.rediris.es/archives/es-tex.html</a>

\end{lstlisting}

\subsection{Día 17 20:20:40, Pablo Rodríguez}

\begin{lstlisting}
On 02/17/2017 07:52 PM, eric wrote:
> hola, no se si te sirve, pero probe mupdf (1.9a+ds1-2) que es la version 
> mas nueva que hay en debian testing (stretch) y tanto "f" como "r" 
> funcionan.
Gracias Eric, pero ¿te escribe en la búsqueda (con '/') alguna letra
acentuada?
Saludos,
Pablo
-- 
<a href="http://www.ousia.tk" target="_blank">http://www.ousia.tk</a>
Archivos de ES-TEX: <a href="http://listserv.rediris.es/archives/es-tex.html" target="_blank">http://listserv.rediris.es/archives/es-tex.html</a>

\end{lstlisting}

\subsection{Día 17 20:56:48, eric}

\begin{lstlisting}
tienes razon, no permite caracteres acentuados en la busqueda ....
On 02/17/2017 04:20 PM, Pablo Rodríguez wrote:
> On 02/17/2017 07:52 PM, eric wrote:
> > hola, no se si te sirve, pero probe mupdf (1.9a+ds1-2) que es la version
> > mas nueva que hay en debian testing (stretch) y tanto "f" como "r"
> > funcionan.
> ...(texto omitido)...
-- 
Forest Engineer
Master in Environmental and Natural Resource Economics
Ph.D. student in Sciences of Natural Resources at La Frontera University
Member in AguaDeTemu2030, citizen movement for Temuco with green city standards for living
Nota: Las tildes se han omitido para asegurar compatibilidad con algunos lectores de correo.
Archivos de ES-TEX: <a href="http://listserv.rediris.es/archives/es-tex.html" target="_blank">http://listserv.rediris.es/archives/es-tex.html</a>

\end{lstlisting}

\subsection{Día 17 23:44:03, Olmo}

\begin{lstlisting}
Hola, a mi mupdf-gl (versi�n 1.10-a) no me deja usar las teclas de f y r,
ciertamente, pero en cuanto a las cuestiones de los otros colisteros, si que
puedo buscar texto con tildes, sin problema (con / ) y me muestra
las tildes cuando las escribo en la b�squeda.
Saludos.
Archivos de ES-TEX: <a href="http://listserv.rediris.es/archives/es-tex.html" target="_blank">http://listserv.rediris.es/archives/es-tex.html</a>

\end{lstlisting}

\subsection{Día 18 14:14:20, Pablo Rodríguez}

\begin{lstlisting}
On 02/17/2017 11:44 PM, Olmo wrote:
> Hola, a mi mupdf-gl (versión 1.10-a) no me deja usar las teclas de f y r,
> ciertamente, pero en cuanto a las cuestiones de los otros colisteros, si que
> puedo buscar texto con tildes, sin problema (con / ) y me muestra
> las tildes cuando las escribo en la búsqueda.
Muchas gracias, Olmo.
Acabo de compilar la última versión de git y tampoco funcionan las teclas.
Por cierto, ¿en qué ha sido: Windows o Linux?
Muchas gracias,
Pablo
-- 
<a href="http://www.ousia.tk" target="_blank">http://www.ousia.tk</a>

\end{lstlisting}

\subsection{Día 18 15:09:52, Pablo Rodríguez}

\begin{lstlisting}
On 02/18/2017 02:14 PM, Pablo Rodríguez wrote:
> On 02/17/2017 11:44 PM, Olmo wrote:
>> Hola, a mi mupdf-gl (versión 1.10-a) no me deja usar las teclas de f y r,
>> ciertamente, pero en cuanto a las cuestiones de los otros colisteros, si que
>> puedo buscar texto con tildes, sin problema (con / ) y me muestra
> ...(texto omitido)...
Me corrijo: 'r' recarga el documento en la última versión de git, pero
'f' no pone a pantalla completa.
Pablo
-- 
<a href="http://www.ousia.tk" target="_blank">http://www.ousia.tk</a>

\end{lstlisting}

\subsection{Día 18 15:39:09, Olmo}

\begin{lstlisting}
En linux, gentoo 64 bits, lo de r, cierto, acabo de probar y
funciona, prob� a compilar el paquete con soporte javascript
tambi�n, por si las moscas, y nada, lo de la pantalla completa
sigue igual, pero en mupdf x11, que era el que suelo usar,
no da problemas.
Saludos,
Olmo
Normas para el correcto uso del correo electr�nico:

\end{lstlisting}
\section{Estudio científico: LaTeX es menos productivo}

\subsection{Día 19 15:15:58, Rubén Gómez Antolí}

\begin{lstlisting}
<a href="http://softlibre.barrapunto.com/softlibre/17/01/29/2325202.shtml" target="_blank">http://softlibre.barrapunto.com/softlibre/17/01/29/2325202.shtml</a>
Ningún ánimo en crear una guerra dialéctica, solo por difundir y recabar
opiniones.
Como usuario de LaTeX, creo que no esta solo el asunto en la
productividad pero...
De hecho, he expuesto algunas de mis razones a favor de *TeX en el foro
del Hacklab Almería:
<a href="https://foro.hacklabalmeria.net/t/wysiwyg-vs-wysiwym-latex-es-menos-productivo/8185" target="_blank">https://foro.hacklabalmeria.net/t/wysiwyg-vs-wysiwym-latex-es-menos-productivo/8185</a>
Salud y Revolución.
Lobo.
-- 
Libertad es poder elegir en cualquier momento. Ahora yo elijo GNU/Linux,
para no atar mis manos con las cadenas del soft propietario.
Porque la libertad no es tu derecho, es tu responsabilidad.
<a href="http://www.mucharuina.com" target="_blank">http://www.mucharuina.com</a>
Desde El Ejido, en Almería, usuario registrado Linux #294013
<a href="http://www.counter.li.org" target="_blank">http://www.counter.li.org</a>
Para darse de baja ES-TEX pincha y envia el siguiente url
mailto:[conectar para ver]

\end{lstlisting}

\subsection{Día 19 22:25:14, Pablo Rodríguez}

\begin{lstlisting}
On 02/19/2017 03:15 PM, Rubén Gómez Antolí wrote:
> <a href="http://softlibre.barrapunto.com/softlibre/17/01/29/2325202.shtml" target="_blank">http://softlibre.barrapunto.com/softlibre/17/01/29/2325202.shtml</a>
> 
> Ningún ánimo en crear una guerra dialéctica, solo por difundir y recabar
> opiniones.
> ...(texto omitido)...
Lobo,
gracias por la referencia. Hace poco la discutía en otro foro.
Yo sí quiero discutir: LaTeX no es productivo en absoluto. Es irónico:
ni quiero polemizar y más o menos productivas somos las personas ;-).
Ya en serio, creo que hay varias cuestiones. Antes de nada, una
declaración para que puedan entenderse mis palabras.
No soy de ciencias, toda mi matemática es digital (sólo sé contar con
los dedos ;-)). He usado LaTeX una década y le estoy muy agradecido a
toda la gente que lo hace posible. Desde que uso ConTeXt, creo que el
cambio ha sido para mejor (dejémoslo así de ambiguo). Y con Markdown en
pandoc, escribir es mucho más fácil.
No todo lo hago con Markdown y pandoc. Uso ConTeXt también en el
trabajo, donde tenemos un sistema automatizado de creación de
documentos. Creo que el ordenador es un asistente digital al que hay que
ordenar con palabras, sólo señalando con un puntero es precario para
muchas cosas. Esto se aplica a la composición tipográfica de textos.
Dicho esto, voy a empezar con una generalización. Es falsa aplicada a
todo el mundo, pero es probable que sea cierta para la mayoría de las
personas. LaTeX es demasiado quizá para la mayoría de gente que lo usa.
No es que la gente no lo deba usar, sino que le sobra la mayoría de las
cosas se les quedan grande. Me explico.
LaTeX tiene una ventaja básica respecto a otras soluciones para generar
documentos PDF: composición matemática automática fácil y de alta
calidad. El resto de soluciones, al menos hace tiempo, eran mucho peores
y más complicadas. La notación matemática es además muy cómoda. Por eso
es difícil dejar de usar LaTeX para mucha gente, porque necesita
fórmulas correctas.
Hay dos detalles en que se ve que TeX es una bestia que se nos queda
grande a la gran mayoría. ¿Cuántos documentos tienen las familias
tipográficas “Computer Modern”? Personalmente pienso que son tipografías
muy pensadas, pero no sé si son excelentes en cuanto a su legibilidad
(les falta cuerpo).
Lo siguiente es un detalle, pero muestra bastante. Creo que es por ser
la opción predeterminada de fancyhdr (si no recuerdo mal, yo lo tuve que
resolver también), hay por ahí varios documentos PDF con encabezados y
pies en páginas en blanco. Con todos los respetos, además de una faena,
eso es una chapuza de mucho cuidado.
LaTeX es problemático porque hace que el autor tenga que ser también el
tipógrafo. Son tareas distintas, pero en si usas un programa de
composición tipográfica, no las puedes separar. De hecho, como ya me
habéis leído aquí, el problema está en que LaTeX no puede separar entre
contenido y presentación, o entre texto y formato. Cualquier archivo de
origen de TeX está contiene órdenes de composición tipográfica.
LaTeX está bien para lo que tiene que hacer. El error es pensar que es
el modo de trabajar informáticamente con textos.
No voy a ser yo quien proponga un procesador de textos. Creo que son un
error grave, porque también mezclan escritura y composición tipográfica.
Y además, de un modo visual, con lo que es mucho más difícil de deshacer
el cacao mental.
La idea es un código de marcación ligero, tipo Markdown. Que lo procese
pandoc y LaTeX hoy, y ya veremos si lo hacemos de otro modo. El problema
está en confundir herramientas y formatos. No deberían tener que ver.
Os voy a poner un ejemplo, en el que creo que coincidiremos. Imaginemos
que tenemos que hacer la composición tipográfica de la tesis de una
persona cercana. En un mundo ideal (o irreal, ya no sé ;-)), la persona
trabajaría con LaTeX. Luego viene la realidad y para distribuirlo como
ePub te las ves y te las deseas.
La solución es coger el texto totalmente redactado y componerlo. Es un
compromiso precario, porque después quien no puede trabajar el texto es
la persona a quienes ayudamos.
¿Un intermedio? Tú trabaja en Markdown y yo ya veré que hago. Claro, por
supuesto que hay que aprender. Pero en una semana sabe bien cualquiera.
En el fondo es volver a distinguir (y separar en muchos casos) la tarea.
El manuscrito digital para quien escribe y la salida electrónica que más
convenga para quien lo tenga que hacer.
Si no somos capaces de distinguir, creo que estamos complicando
innecesariamente las cosas.
Pero como todo, puede ser que me equivoque de cabo a rabo. Por eso,
discutamos (que no riñamos, o siquiera polemicemos, sobre) el asunto.
Saludos,
Pablo
-- 
<a href="http://www.ousia.tk" target="_blank">http://www.ousia.tk</a>
Para darse de baja ES-TEX pincha y envia el siguiente url
mailto:[conectar para ver]

\end{lstlisting}

\subsection{Día 19 23:10:02, eric}

\begin{lstlisting}
Hola, revise el articulo de la noticia (no exhaustivamente, claro) y me 
deja varias inquietudes como "articulo cientifico":
1. Con relacion a la metodologia ... en que dimension o universo 
paralelo la cantidad de errores ortograficos y gramaticales podria estar 
mas relacionado con el software de edicion/formateo de texto que con la 
persona que escribe.
2. La cantidad de texto producida solo esta indirectamente relacionada 
con el software de edicion que se use. Esa variable creo que esta mucho 
mas influenciada por la practica de la persona que teclea.
3. Mucho mas que Latex, la GUI que se use para producir el texto 
influira mucho mas en la cantidad de tablas y ecuaiones que se puedan 
escribir. Por ejemplo, escribir a mano el codigo para tablas es una 
joda, pero hay GUI como textstudio que facilitan todo ese trabajo. Se 
configura/inserta el codigo en 10 segundos y luego solo te dedicas a 
rellenar la tabla ... si en cambio lo haces a mano, evidentemente que 
vas a producir menos texto que insertando una tabla en word.
4. Revise las referencias y para empezar es super pobre, son poquisimas 
referencias para un articulo "cientifico". Los enlaces de la primera 
referencia no conducen a ningun documento valido relacionado con el 
tema. Hay 4 textos que son de usabilidad, lo que es apenas un pequeño 
aspecto de lo que deberia ser el estudio. Hay un articulo del año 1957 
!!!. Y la ultima referencia de la OCDE no puedo encontrarla.
No tengo tiempo de mirar mas en detalle ese "estudio" pero asi por 
encima dudo mucho de su calidad, conclusiones y recomendaciones.
Slds.
On 02/19/2017 11:15 AM, Rubén Gómez Antolí wrote:
> <a href="http://softlibre.barrapunto.com/softlibre/17/01/29/2325202.shtml" target="_blank">http://softlibre.barrapunto.com/softlibre/17/01/29/2325202.shtml</a>
>
> Ningún ánimo en crear una guerra dialéctica, solo por difundir y recabar
> opiniones.
> ...(texto omitido)...
-- 
Forest Engineer
Master in Environmental and Natural Resource Economics
Ph.D. student in Sciences of Natural Resources at La Frontera University
Member in AguaDeTemu2030, citizen movement for Temuco with green city 
standards for living
Nota: Las tildes se han omitido para asegurar compatibilidad con algunos 
lectores de correo.
Para darse de baja ES-TEX pincha y envia el siguiente url
mailto:[conectar para ver]

\end{lstlisting}

\subsection{Día 19 23:40:24, Santiago Higuera}

\begin{lstlisting}
Hola:
En mi caso, cada vez que escribo algo que no sea una simple nota con un 
procesador tipo Word, Libre Office o similar, acabo de los nervios. Me 
encanta escribir en Latex, aunque reconozco que a veces hay que 
consultar cosas para encontrar la manera de cuadrar lo que uno quiere 
hacer. A día de hoy, soy incapaz de escribir un documento complejo 
utilizando un procesador tipo Word.
Este hilo me ha traído a la memoria una charla a la que asistí hace un 
par de meses en la Librería Náutica Robinsón. Se trataba de la 
presentación de la nueva edición del libro de Navegación Astronómica de 
Luis Mederos, el mejor libro en su clase, según mi opinión. Al final de 
la charla hablaba de los motivos por los que hay personas que utilizan 
el sextante. Decía que, hoy en día, en un pequeño barco de recreo 
facílmente van seis o siete GPS embarcados. Hay gente que defiende que 
hay que saber utilizar el sextante, en navegación de altura, por si 
falla el GPS. Luis afirmaba que hay menos probabilidades de que falle el 
GPS que de que se caiga el sextante, se rompa un espejo y quede 
inutilizado. Al final decía: el sextante hay que utilizarlo porque es 
divertido y gratificante hacerlo. Permite profundizar en el conocimiento 
de los astros y su movimiento en el cielo. Algo parecido sucede cuando 
uno realiza cálculos con las antiguas reglas de cálculo, a las que 
también soy aficionado. No se trata de ir más rápido que con una 
calculadora electrónica, que a veces también sucede. Se trata de que uno 
se ve obligado a dominar la resolución del problema, el orden de 
magnitud del resultado antes de hacer la operación. Y eso da mucho 
conocimiento y gimnasia mental.
En relación con Latex, en mi caso sin duda soy más eficiente escribiendo 
con Latex que con un procesador. Pero además, me resulta muy 
gratificante. Me divierte. Y eso, yo creo que es suficiente motivo, no 
necesito buscar más motivos.
Un saludo
Santiago Higuera
El 19/02/17 a las 23:10, eric escribió:
> Hola, revise el articulo de la noticia (no exhaustivamente, claro) y 
> me deja varias inquietudes como "articulo cientifico":
>
> 1. Con relacion a la metodologia ... en que dimension o universo 
> ...(texto omitido)...
Para darse de baja ES-TEX pincha y envia el siguiente url
mailto:[conectar para ver]

\end{lstlisting}

\subsection{Día 20 02:33:50, alberto alejandro moyano}

\begin{lstlisting}
Hola,
Creo que como en la vida las cosas no son blancos y negros, es decir, hay
grises --y creo que muchos--, yo utilizo LaTeX para producir libros para mi
editorial desde hace más de 12 años, quiero decir, en todos estos años ya
llevo más de 400 libros editados/armados con un promedio de 256 páginas
cada uno (para ponerlo bien en claro, son algo más de 100.000 páginas);
durante algunos años fui beta tester de Latinoamérica para Adobe para su
producto Page Maker 6.5/7.0 (el antecesor de InDesign), incluso hoy cada
tanto dedico algo de tiempo a ver en que anda InDesign para ver si arrima
el bochín a la performance/potencia de edición/armado de LaTeX, pero --aún
viendo mejoras-- solo puedo confirmar que a LaTeX no hay con que darle, al
menos en mi sistema de edición.
Ergo, en mi caso --una pequeña editorial (35/37 títulos por año)--,
discutir la eficiencia de editar usando LaTeX en comparación con los
sistemas convencionales, ya sea con relación a los tiempos de producción o
a su capacidad, es algo que no tiene sentido.
Saludos,
Alberto
================
El 19 de febrero de 2017, 19:40, Santiago Higuera <<a href="/cgi-bin/wa?LOGON=A3%3Dind1702%26L%3DES-TEX%26E%3Dquoted-printable%26P%3D246575%26B%3D--001a113e1f5477ac850548ec414f%26T%3Dtext%252Fplain%3B%2520charset%3DUTF-8%26header%3D1" target="_parent" >[conectar para ver]</a>>
escribió:
> Hola:
>
> En mi caso, cada vez que escribo algo que no sea una simple nota con un
> procesador tipo Word, Libre Office o similar, acabo de los nervios. Me
> ...(texto omitido)...
Si tiene algun problema con la utilizacion de la lista.
Pongase en contacto con nosotros a traves de:
<a href="/cgi-bin/wa?LOGON=A3%3Dind1702%26L%3DES-TEX%26E%3Dquoted-printable%26P%3D246575%26B%3D--001a113e1f5477ac850548ec414f%26T%3Dtext%252Fplain%3B%2520charset%3DUTF-8%26header%3D1" target="_parent" >[conectar para ver]</a>

\end{lstlisting}

\subsection{Día 20 02:55:16, Juan Manuel Macías}

\begin{lstlisting}
Hola, completamente de acuerdo contigo, Alberto. Yo ya llevo también unos
cuantos libros (¡no tantos como tú!) maquetados y compuestos en LaTeX. O,
para ser más precisos, en TeX y sus formatos... Y a día de hoy no tiene
rival en la confección de cualquier libro, ya sea científico, de poesía,
una simple novela, un diccionario, etc. InDesign, siendo un gran programa,
juega en otra liga: magacines ilustrados, periódicos, etc. El gran problema
es que se ha convertido en un estándar casi de facto en la producción
editorial, sobre todo por culpa de lo conformista y anquilosada que está
esta industria, ya de antiguo. Pero eso es otro cantar. Por principios
personales respecto al software privativo, no uso InDesign sino Scribus,
para hacer cubiertas de libros casi exclusivamente. No es tan refinado como
InDesign, pero va evolucionando de manera muy prometedora.
En cuanto a lo del estudio de marras, poco que añadir a lo que se ha
comentado aquí. Tengo muy poca fe en este tipo de estudios tan
"mediáticos". Son muchos los casos y contextos posibles, si se usa LaTeX
como un mero procesador de texto frente a los procesadores WYSIWYG. Yo no
lo uso en la escritura de a diario, por ejemplo: más bien suelo escribir en
texto plano, o con algo de markdown, como Pablo, pero es mi caso personal e
intransferible. Creo que este estudio pasa por alto muchas cosas a la
ligera y parte de una concepción un tanto frívola de la "productividad".
Saludos,
Juan Manuel
El 20 de febrero de 2017, 2:33, alberto alejandro moyano <
<a href="/cgi-bin/wa?LOGON=A3%3Dind1702%26L%3DES-TEX%26E%3Dquoted-printable%26P%3D268173%26B%3D--001a1146930414b7c40548ec8e94%26T%3Dtext%252Fplain%3B%2520charset%3DUTF-8%26header%3D1" target="_parent" >[conectar para ver]</a>> escribió:
> Hola,
>
> Creo que como en la vida las cosas no son blancos y negros, es decir, hay
> grises --y creo que muchos--, yo utilizo LaTeX para producir libros para mi
> ...(texto omitido)...
Si tiene algun problema con la utilizacion de la lista.
Pongase en contacto con nosotros a traves de:
<a href="/cgi-bin/wa?LOGON=A3%3Dind1702%26L%3DES-TEX%26E%3Dquoted-printable%26P%3D268173%26B%3D--001a1146930414b7c40548ec8e94%26T%3Dtext%252Fplain%3B%2520charset%3DUTF-8%26header%3D1" target="_parent" >[conectar para ver]</a>

\end{lstlisting}

\subsection{Día 20 16:47:47, Javier Bezos}

\begin{lstlisting}
El 19/02/2017 15:15, Rubén Gómez Antolí escribió:
> <a href="http://softlibre.barrapunto.com/softlibre/17/01/29/2325202.shtml" target="_blank">http://softlibre.barrapunto.com/softlibre/17/01/29/2325202.shtml</a>
>
> Ningún ánimo en crear una guerra dialéctica, solo por difundir y recabar
> opiniones.
Dejémoslo en estudio. De científico tiene poco. Los errores
metodológicos y de concepto ya se han expuesto en muchos foros
(porque ya tiene unos cuantos años).
Chao
Javier
Si tiene algun problema con la utilizacion de la lista.
Pongase en contacto con nosotros a traves de:
<a href="/cgi-bin/wa?LOGON=A2%3Dind1702%26L%3DES-TEX%26F%3D%26S%3D%26P%3D19779" target="_parent" >[conectar para ver]</a>

\end{lstlisting}

\subsection{Día 21 00:37:56, Aradenatorix Veckhôm Awecaelus}

\begin{lstlisting}
Hola a todos:
Pues concuerdo con las opiniones de la mayoría de quienes se me han
adelantado en sus comentarios. De hecho Javier ha dado en el clavo:
cuando vi el titulo de este hilo lo primero que me vino a la mente fue
el artículo de finales de 2014 que comentamos a principios de 2015 en
esta lista sobre este curioso estudio cuya metodología deja mucho que
desear como ya eric lo expuso arriba.
Mi impresión sobre la publicación de barrapunto que sin ahondar
resucita el mentado artículo es venir a picar la cresta o joder para
hacer ruido y ganar algo de tráfico, no le daría más importancia al
asunto.
Yo concuerdo con Pablo en que los procesadores de texto son
herramientas muy malas porque confunden dos tareas distintas que
terminan por salir mal la mayoría de las veces que es escribir
(redactar, mecanografiar, capturar o como le digan) un texto y darle
formato visual a ese texto para que aparezca compuesto
tipográficamente de una forma en que pueda leerse y cumpla con ciertos
estándares del medio que lo publique, etc.
La parte en que difiero con Pablo es que LaTeX confunda las cosas
igualmente que Word. Como sabemos, es posible armar un documento
maestro e ir insertando el contenido con \include sin mayor problema,
algo similar sucede con ConTeXt. Incluso esta posibilidad de componer
texto en markdown y exportarlo a pandoc para generar un archivo .tex
funciona de una manera similar, ya que pandoc se encarga de darle un
preámbulo a nuestro documento a partir de una plantilla que ya sea la
que trae por defecto o una que hagamos nosotros y eso es bastante
útil.
Aprovecho el punto para no dejar en el tintero un error (bug) presente
en las cases estándar que como bien dice Pablo, termina siendo una
chapuza: me refiero a cuando usamos la clase book con la opción por
defecto openright (o report con openright en vez de openany) y la
página que antecede al inicio de un capítulo queda impresa con el
encabezado del anterior en vez de ser totalmente blanca. Ya en las FAQ
del CervanTeX desde hace años se ha publicado un fragmento de código
que lo corrige, adicionalmente si en vez de usar las clases estándar
usamos KOMA Script o Tufte LaTeX, dicho error desaparece, el problema
es que son pocos los usuarios que se adentran más allá de las opciones
predeterminadas, tanto en las clases como en la tipografía a emplear.
Esa es la causa que la ultima impresión del Manifiesto Telecomunista
haya sido impreso de forma bastante anodina por más consejos que al
respecto hicimos algunos a quien se encargó de generar el pdf (con
pandoc) pasando por LaTeX.
Ahí los que no solamente usamos sino que disfrutamos de usar LaTeX
como Santiago acertadamente mencionó, podríamos haberle puesto un poco
más de cuidado a esos detalles que para la mayoría son casi invisibles
o que carecen de interés.
Pero lo cierto es que es difícil que las masas se adentren a LaTeX y
lo adopten con el entusiasmos de los que confluimos en esta lista de
correo y tal vez otros espacios. Yo lo he intentado en mi trabajo y
fracasé estrepitosamente, creo que debí hacer caso y acercarme a
MarkDown antes, ahora creo que es un punto intermedio que permite
generar documentos bien hechos sin necesidad de aprender tanto como
cualquier encarnación de TeX demanda, y tal vez sea más que suficiente
para las necesidades de una buena parte de la población, para los
demás puede servir como punto de partida para adentrarse a LaTeX, al
menos esa es la idea que tengo ahora que me han propuesto dar un
taller para tesistas.
Y esa ha sido parte de la experiencia en la Campechana Mental
(<a href="https://campechana.nomia.mx" target="_blank">https://campechana.nomia.mx</a>) donde hemos buscado rutas para publicar
y digitalizar textos, y markdown hasta ahora viene siendo el punto
neutro desde el cual podemos hacer estas cosas. No desdeño para nada
las bondades de TeX que quedan manifiestas frente a ePub al poder
automatizar o resolver problemas que en ePub algunos intentan resolver
mediante scripts como son el uso de versalitas, la numeración
automática de las notas o el poder hacer ePubs con partituras y otras
cuestiones que van saliendo sobre la marcha.
Todavía no tengo el gusto de publicar libros compuestos con LaTeX como
Alberto, tampoco tengo una editorial para ello, pero es interesante
ver como incluso en colectivos que gustan de editar y publicar libros
y textos en general, todavía el uso de sistemas de marcado les parece
algo lejano o de ciencia ficción y siguen usando Word, Writer e
InDesign para poder resolver sus necesidades de publicación, o Scribus
en el mejor de los casos por ser software libre. Incluso la vez que se
platicó del tema con Traficantes de Sueños se quedaron azorados por
las posibilidades de la herramienta, pero no tenían forma de modificar
su flujo de trabajo basado en InDesign.
Recuerdo cuando en 2011 yo dije en uno de los mensajes de la lista que
LaTeX era para cuestiones técnicas y para lo demás InDesign *risas
grabadas*. Ahora creo que InDesign y Scribus son una buena opción para
componer revistas, periódicos y publicaciones periódicas que sean,
pero intento hacer de todo  en LaTeX y cuando quiero obtener una
salida para imprenta no he hallado nada mejor que TeX.
Saludos
Los artículos de ES-TEX son distribuidos gracias al apoyo y colaboración 
técnica de RedIRIS - Red Académica española - (<a href="http://www.rediris.es" target="_blank">http://www.rediris.es</a>)

\end{lstlisting}

\subsection{Día 21 00:51:25, Juan Manuel Macías}

\begin{lstlisting}
InDesign (o Scribus) hacen muy bien aquello para lo que fueron 
diseñados: maquetar, es decir, diagramar las páginas. Ven la página 
"desde lejos", por decirlo de alguna forma. Esto en revistas ilustradas 
(magacines) y periódicos es un avance. Pero en los libros la maquetación 
tiende a ser más elemental, más estable: cabeceras, pies, caja de texto 
y márgenes. Las batallas se dan más bien en la composición tipográfica y 
es una batalla muchas veces cuerpo a cuerpo. Por eso TeX y sus formatos 
son la herramienta ideal para componer libros, del tipo que sean, no 
sólo científicos: ése es el dichoso sambenito que hay que empezar a 
extirpar (en mi opinión) del mundo de TeX. Sin ir más lejos, hace poco 
la autora de un libro me pidió (se lo estoy maquetando, es una 
traducción al griego del Quijote) si no sería muy descabellado que las 
notas al pie (unas 1700 notas) fueran en formato párrafo, una seguida de 
otra. Esto, naturalmente, sólo se puede hacer con TeX & family. Opté por 
la solución que da el macropaquete reledmac para LaTeX, auqnue también 
es posible hacer algo parecido con footmisc.
Saludos,
JM
El 21/02/17 a las 00:37, Aradenatorix Veckhôm Awecaelus escribió:
> Hola a todos:
>
> Pues concuerdo con las opiniones de la mayoría de quienes se me han
> adelantado en sus comentarios. De hecho Javier ha dado en el clavo:
> ...(texto omitido)...
Los artículos de ES-TEX son distribuidos gracias al apoyo y colaboración 
técnica de RedIRIS - Red Académica española - (http://www.rediris.es)

\end{lstlisting}

\subsection{Día 21 01:37:49, Aradenatorix Veckhôm Awecaelus}

\begin{lstlisting}
Muy interesante trabajo el que haces con reledmac y concuerdo en que
maquetaciones más simples o estables salen bien con prácticamente
cualquier encarnación de TeX. Aunque Nicola Talbot creo flowfram para
no tener que ir a Scribus o InDesign y poder maquetar
(paramétricamente) dentro de LaTeX.
Saludos
Los artículos de ES-TEX son distribuidos gracias al apoyo y colaboración 
técnica de RedIRIS - Red Académica española - (<a href="http://www.rediris.es" target="_blank">http://www.rediris.es</a>)

\end{lstlisting}

\subsection{Día 21 01:57:22, Juan Manuel Macías}

\begin{lstlisting}
El paquete flowfram me parece un trabajo encomiable como "teoría" pero, en
mi humilde opinión, lo veo muy poco práctico. Ahí sí que LaTeX estaría
detrás de los programas DTP, sería el caso inverso a lo que comenté antes
de las notas y reledmac. En un caso u otro, ¿por qué no usar siempre lo más
sencillo, rápido y efectivo? En unos casos será (La)TeX y en otros Scribus
o InDesign: una interfaz para cada contexto, y creo que ahí es donde reside
realmente la productividad en la que tanto insistía el pseudoestudio ese
que comentamos en este hilo.
Saludos,
JM
El 21 de febrero de 2017, 1:37, Aradenatorix Veckhôm Awecaelus <
<a href="/cgi-bin/wa?LOGON=A3%3Dind1702%26L%3DES-TEX%26E%3Dquoted-printable%26P%3D353815%26B%3D--f403045f532cdad4110548ffdcc6%26T%3Dtext%252Fplain%3B%2520charset%3Dutf-8%26header%3D1" target="_parent" >[conectar para ver]</a>> escribió:
> Muy interesante trabajo el que haces con reledmac y concuerdo en que
> maquetaciones más simples o estables salen bien con prácticamente
> cualquier encarnación de TeX. Aunque Nicola Talbot creo flowfram para
> no tener que ir a Scribus o InDesign y poder maquetar
> ...(texto omitido)...
Los artículos de ES-TEX son distribuidos gracias al apoyo y colaboración 
técnica de RedIRIS - Red Académica española - (http://www.rediris.es)

\end{lstlisting}

\subsection{Día 21 02:18:34, Aradenatorix Veckhôm Awecaelus}

\begin{lstlisting}
El 20 de febrero de 2017, 18:57, Juan Manuel Macías
<<a href="/cgi-bin/wa?LOGON=A2%3Dind1702%26L%3DES-TEX%26F%3D%26S%3D%26P%3D24212" target="_parent" >[conectar para ver]</a>> escribió:
> El paquete flowfram me parece un trabajo encomiable como "teoría" pero, en mi humilde opinión, lo veo muy poco práctico.
Concuerdo parcialmente con este comentario, mencioné al paquete
flowfram porque es un desarrollo interesante que amplía las
posibilidades de uso de LaTeX, ya de por sí bastante grandes a un
nicho que se considera exclusivo de los DTP como bien mencionas. Y en
ese sentido creo que para un proyecto pequeño puede ser util, pero no
me imagino maquetar todo un diario o una revista completa en LaTeX con
flowfram, algo que comentaba con un amigo la semana pasada. En todos
caso, como usuario de software libre diría que sale más a cuenta
hacerlo en Scribus con la añadidura de que admite el uso de LaTeX, no
he explorado esa posibilidad para decir de cierto como y hasta donde,
intuyo que será para dar consistencia a formatos de cebeceras y a
ciertas caracterísitcas de párrafos.
> Ahí sí que LaTeX estaría detrás de los programas DTP, sería el caso inverso a lo que comenté antes de
> las notas y reledmac. En un caso u otro, ¿por qué no usar siempre lo más sencillo, rápido y efectivo?
Yo creo que esa es la idea, aunque cuando nos metemos con un DTP no
siempre es fácil recuperar los contenidos y poder reeditarlos
fácilmente lo que provoca muchas veces que lo más sencillo, rápido y
efectivo a corto plazo, no lo sea pensando en algo menos inmediato,
aunque esa me parece que es una discusión más profunda que excede por
mucho al tema de hoy.
> En unos casos será (La)TeX y en otros Scribus o InDesign: una interfaz para cada contexto, y creo que ahí es donde reside
> realmente la productividad en la que tanto insistía el pseudoestudio ese que comentamos en este hilo.
Más que la productividad, yo creo que sería cierta facilidad, no se me
ocurre una categoría más precisa, actualmente contamos con tantas
opciones como para seguir quebrandonos la cabeza y, por poner un par
de ejemplos burdos, componer textos en procesadores de texto o hacer
libros con herramientas de trazado vectorial en vez de un DTP como
sigue ocurriendo.
Saludos
Los artículos de ES-TEX son distribuidos gracias al apoyo y colaboración 
técnica de RedIRIS - Red Académica española - (<a href="http://www.rediris.es" target="_blank">http://www.rediris.es</a>)

\end{lstlisting}

\subsection{Día 21 02:36:57, Juan Manuel Macías}

\begin{lstlisting}
Cierto, facilidad, pero siempre en términos relativos. Por ejemplo, se ha
generalizado la idea de que "lo fácil" es una interfaz gráfica y eso es
aplicable sólo en unos casos. Lo fácil sería apretar un botón y que se
barra la casa, pero no apretar un botón y tener que seguir a la escoba en
todo momento, porque para eso, me quedo con la escoba y dejo el botón :-)
Componer una edición crítica en InDesign o Scribus no es que sea imposible,
pero acabaría resultando absurdo, un dolor, una tortura china. Aunque el
proceso en sí, para lo que entiende InDesign ("mueve, coloca, mueve,
coloca") no tiene ninguna complicación. Pero, para eso, mejor usar una
imprenta mecánica de las de antes.
Saludos,
JM
El 21 de febrero de 2017, 2:18, Aradenatorix Veckhôm Awecaelus <
<a href="/cgi-bin/wa?LOGON=A3%3Dind1702%26L%3DES-TEX%26E%3Dquoted-printable%26P%3D375617%26B%3D--001a11469304731ae10549006a4f%26T%3Dtext%252Fplain%3B%2520charset%3Dutf-8%26header%3D1" target="_parent" >[conectar para ver]</a>> escribió:
> El 20 de febrero de 2017, 18:57, Juan Manuel Macías
> <<a href="/cgi-bin/wa?LOGON=A3%3Dind1702%26L%3DES-TEX%26E%3Dquoted-printable%26P%3D375617%26B%3D--001a11469304731ae10549006a4f%26T%3Dtext%252Fplain%3B%2520charset%3Dutf-8%26header%3D1" target="_parent" >[conectar para ver]</a>> escribió:
> > El paquete flowfram me parece un trabajo encomiable como "teoría" pero,
> en mi humilde opinión, lo veo muy poco práctico.
> ...(texto omitido)...
Los artículos de ES-TEX son distribuidos gracias al apoyo y colaboración 
técnica de RedIRIS - Red Académica española - (http://www.rediris.es)

\end{lstlisting}

\subsection{Día 21 11:38:59, Rubén Gómez Antolí}

\begin{lstlisting}
El 21/02/17 a las 00:37, Aradenatorix Veckhôm Awecaelus escribió:
[...]
> Hola a todos:
> 
> Pues concuerdo con las opiniones de la mayoría de quienes se me han
> adelantado en sus comentarios. De hecho Javier ha dado en el clavo:
> ...(texto omitido)...
Pido perdón, pues, había olvidado que ya se trató aquí.
Lo de «científico» se me escapó, me pareció haberlo leído en algún sitio.
Y quiero aclarar que también albergaba dudas sobre la metodología
utilizada en dicho estudio.
Como decía, mi única intención era exponerlo y poder debatirlo si se
creía interesante, como así ha sucedido y, como siempre sucede en esta
lista, por mi parte algo he aprendido.
Gracias a todos.
Salud y Revolución.
Lobo.
-- 
Libertad es poder elegir en cualquier momento. Ahora yo elijo GNU/Linux,
para no atar mis manos con las cadenas del soft propietario.
Porque la libertad no es tu derecho, es tu responsabilidad.
<a href="http://www.mucharuina.com" target="_blank">http://www.mucharuina.com</a>
Desde El Ejido, en Almería, usuario registrado Linux #294013
<a href="http://www.counter.li.org" target="_blank">http://www.counter.li.org</a>
Los artículos de ES-TEX son distribuidos gracias al apoyo y colaboración 
técnica de RedIRIS - Red Académica española - (<a href="http://www.rediris.es" target="_blank">http://www.rediris.es</a>)

\end{lstlisting}

\subsection{Día 25 16:05:53, Pablo Rodríguez}

\begin{lstlisting}
On 02/21/2017 12:37 AM, Aradenatorix Veckhôm Awecaelus wrote:
> Hola a todos:
Hola Aradnix,
siento no haber contestado antes. Durante la semana tengo menos tiempo
disponible para estos menesteres.
Por tus alusiones contesto, aunque quiero hacer antes algunas aclaraciones.
No he leído el artículo original, ni lo pienso leer. No sé si es bueno o
no (aunque es probable que sea bastante malo). Sin embargo, creo que
tiene sentido afirmar que TeX puede incidir en la productividad.
TeX es «una imprenta en manos», como reza el excelente título. Creo que
la mayoría de la gente no sabría que hacer con esa imprenta. Se les
queda demasiado grande.
Eso no es problema de TeX, ni necesariamente problema de la gente.
Porque no existe la utilidad universal, siempre depende de las
necesidades individuales. Y cuando una herramienta es excesiva, acaba
molestando.
Por supuesto, no voy a negar que TeX haga excelentemente su trabajo.
Alberto y Juan Manuel nos muestran su trabajo diario con TeX. Se dedican
a la composición tipográfica y edición de textos. TeX es la herramienta
adecuada para su tarea.
TeX puede ser excesivo (y creo que lo es en muchos casos) para quien
sólo tiene que generar un documento PDF de sus escritos.
> La parte en que difiero con Pablo es que LaTeX confunda las cosas
> igualmente que Word. Como sabemos, es posible armar un documento
> maestro e ir insertando el contenido con \include sin mayor problema,
> algo similar sucede con ConTeXt. Incluso esta posibilidad de componer
> ...(texto omitido)...
No he dicho que TeX confunda las cosas como un procesador de textos. En
TeX, formato y contenido no pueden estar separados porque TeX sólo
entiende sus órdenes.
Por supuesto, podemos entender "\chapter{}" como un modo de etiquetado.
Pero en realidad es una orden tipográfica. Aunque sólo fuese, porque
sólo TeX la entiende.
Respecto a Markdown, incluirlo en documentos con TeX no exime de la
necesidad de lidiar con TeX. No es que no sea útil, es que no es útil
para todo el mundo.
Usar pandoc como escudo frente a TeX (para escribir en Markdown y
covertirlo a LaTeX) no es la solución mágica. O no es tan flexible como
TeX mismo, o acabas teniendo que mantener dos orígenes (Markdown y TeX).
Mi experiencia es que incluso acaban pasando las dos cosas.
Lo habré mencionado ya, pero lo vuelvo a mencionar. Uno de los mayores
problemas (por supuesto, a mi juicio) que tiene el desarrollo de pandoc
es su anclaje en LaTeX.
A pesar de que su autor tiene (o tuvo [ya no lo sé]) claro que es
absurdo replicar TeX con pandoc, de las incidencias abiertas en GitHub,
creo que una parte no desdeñable son del tipo «cómo hacer x con LaTeX».
Personalmente creo que es un error.
De hecho, me he pensado muy en serio ofrecer una alternativa a LaTeX
para la generación de documentos PDF en pandoc. Sin embargo, intuyo que
habrá un problema insoluble. LaTeX es irremplazable cuando sólo nos
sirve LaTeX. Si las presentaciones en beamer tienen que ser exactamente
iguales, o si todo tiene que funcionar como determinados paquetes de
LaTeX, lo que se está expresando es que no se quiere o no se puede cambiar.
Me temo que esto último es exactamente igual que Traficantes de Sueños
con InDesign. Y no creo que sea una virtud, sino más bien su contrario.
Para que quede claro no me refiero a que la alternativa no haga lo que
hace LaTeX, sino que no lo haga como lo hace LaTeX.
> [...]
> el problema es que son pocos los usuarios que se adentran más allá de
> las opciones predeterminadas, tanto en las clases como en la
> tipografía a emplear.
En ese caso, yo afirmaría que la composición tipográfica no les interesa
a esas personas. Sus intereses son otros y quizá por eso TeX se les
quede grande.
Algo que a mí me da mucho que pensar es lo siguiente. La mayoría de
libros que caen en mis manos no consiguen ligar el par fi o fl. Da lo
mismo la editorial. Si ése es el nivel que tienen quienes viven de eso,
¿qué pretendemos que hagan el resto?
> Pero lo cierto es que es difícil que las masas se adentren a LaTeX y
> lo adopten con el entusiasmos de los que confluimos en esta lista de
> correo y tal vez otros espacios. Yo lo he intentado en mi trabajo y
> fracasé estrepitosamente, creo que debí hacer caso y acercarme a
> ...(texto omitido)...
Creo que ahí hay dos cuestiones distintas: un impedimento y el trabajo
con TeX mismo.
El impedimento es la imagen visual. Si todos los textos que genera una
persona son con un procesador de textos, es muy difícil (o le llevará un
tiempo) acostumbrarse a trabajar archivos de texto puro y compilación o
interpretación.
Respecto a TeX mismo, creo que a la mayoría le interesa el resultado,
pero le supera aprender. En mi trabajo lo usamos a diario, para la
generación automatizada de ciertos documentos. Pero todo está de tal
manera que no se nota nada.
> [...]
> pero intento hacer de todo  en LaTeX y cuando quiero obtener una
> salida para imprenta no he hallado nada mejor que TeX.
Yo también hago lo mismo con ConTeXt. Sin embargo, creo que una cosa es
la ortografía tipográfica (como el uso de «comillas latinas») y otra la
composición tipográfica. Lo primero es exigible a todo el mundo, lo
segundo sólo a quien compone libros.
Saludos,
Pablo
-- 
<a href="http://www.ousia.tk" target="_blank">http://www.ousia.tk</a>
Si tiene algun problema con la utilizacion de la lista.
Pongase en contacto con nosotros a traves de:
<a href="/cgi-bin/wa?LOGON=A2%3Dind1702%26L%3DES-TEX%26F%3D%26S%3D%26P%3D41674" target="_parent" >[conectar para ver]</a>

\end{lstlisting}

\subsection{Día 25 17:59:13, alberto alejandro moyano}

\begin{lstlisting}
> > [...]
> > el problema es que son pocos los usuarios que se adentran más allá de
> > las opciones predeterminadas, tanto en las clases como en la
> > tipografía a emplear.
> ...(texto omitido)...
Creo que acá se ha puesto el dedo en la llaga, porque «pocos usuarios» es
ambiguo, no es el mismo análisis cuando ese universo de usuarios son
finales (léase tesistas, investigadores, etc.) que si son del mundo de la
producción editorial (léase editores, estudios de diseño y armado, etc.).
Una pregunta que me hice en varias oportunidades, es porqué LaTeX no
pudo/puede entrar en el mundo profesional de la edición, se que hay
excepciones, pero sobran dedos de la mano para contar.
Si tiene algun problema con la utilizacion de la lista.
Pongase en contacto con nosotros a traves de:
<a href="/cgi-bin/wa?LOGON=A3%3Dind1702%26L%3DES-TEX%26E%3Dquoted-printable%26P%3D588961%26B%3D--001a113d59741a1e3a05495dc40a%26T%3Dtext%252Fplain%3B%2520charset%3DUTF-8%26header%3D1" target="_parent" >[conectar para ver]</a>

\end{lstlisting}

\subsection{Día 25 18:53:42, Juan Manuel Macías}

\begin{lstlisting}
Conozco un poco el mundo editorial de aquí en España, como autor y como
"maquetador" (un término un tanto laxo pero nos entendemos), y si hay una
palabra que lo define es conformismo. Imagino que es una pandemia. Pablo ha
puesto un ejemplo simple y definitivo. La ligadura fi no la respetaba nadie
o casi nadie en los libros porque el 99 por ciento de la producción
editorial se hacía hasta hace relativamente poco tiempo con Quark Xpress,
un software bastante limitado pero que ha sabido reinar como nadie en esto,
precisamente debido a ese conformismo, y a que nadie exigía más. Porque es
eso: simplemente nadie exigía más. No es que Quark no pudiera crear la
ligadura, es que no lo hacía "de fábrica". Con Quark levantabas una piedra
y todos se autonombraban diseñadores gráficos, una de las profesiones más
prostituidas de los últimos tiempos, pues tengo un gran respeto hacia el
verdadero diseño gráfico, que (aun habiendo ciertos vasos comunicantes)
poco tiene que ver con la tipografía. Quiero decir, los ingredientes eran
muy simples: Quark (software limitado y poco dado a evolucionar) y una
pléyade de usuarios poco interesados en los intríngulis tipográficos, en
manos de la maldición del WYSiWYG. La jugada de Adobe fue inteligente. El
incipiente InDesign implementó los algorismos Plass/knuth de composición de
párrafo originales de TeX y añadió algunas cuantas funcionalidades
microtipográficas, aprovechando que llegaba la revolución de las fuentes
Open Type. En tiempos del InDesign 2.0 ni siquiera Quark soportaba Unicode.
Y a Quark le ocurrió lo que a Nokia, se durmió en sus propios laureles y se
le pasó el arroz. Ahora es InDesign el nuevo amo en la producción
editorial, aunque sus planteamientos siguen siendo igual de limitados que
los de Quark para la confección de libros, como ya hemos comentado aquí.
Eso sí, hoy los libros vienen con un aspecto más aceptable debido,
precisamente, a esas cualidades microtipográficas que InDesign trae de
fábrica, y hasta se ve más a menudo la ligadura fi.
Dentro del "gremio" de la maquetación conozco a personas que se han
interesado honestamente por todo lo que ofrece TeX. Y también los hay que
me han respondido: "¡Pero cómo! ¿Eso no es un programa para hacer fórmulas
matemáticas?". Está claro el sambenito...
En cualquier caso, luchar contra el conformismo en la producción editorial
es difícil, casi tanto como luchar contra cierta endogamia y conformismo
"por reacción". La única baza que tenemos es la de ofrecer resultados. Y es
que sigo pensando que cualquier persona que esté seriamente interesada en
la tipografía (seriamente, subrayo) unida a los avances en el mundo digital
ha de recabar tarde o temprano en TeX; y digo TeX, sin prefijos. Los
formatos son necesarios e imprescindibles, pero hay que darle al César lo
que es del César y poner cada cosa en su lugar. Los desarrolladores de
InDesign sabían dónde rebañar...
Saludos,
Juan Manuel
El 25 de febrero de 2017, 17:59, alberto alejandro moyano <
<a href="/cgi-bin/wa?LOGON=A3%3Dind1702%26L%3DES-TEX%26E%3Dquoted-printable%26P%3D599912%26B%3D--f403045f532cf07c0005495e868b%26T%3Dtext%252Fplain%3B%2520charset%3DUTF-8%26header%3D1" target="_parent" >[conectar para ver]</a>> escribió:
>
> > [...]
>> > el problema es que son pocos los usuarios que se adentran más allá de
>> > las opciones predeterminadas, tanto en las clases como en la
> ...(texto omitido)...
Si tiene algun problema con la utilizacion de la lista.
Pongase en contacto con nosotros a traves de:
<a href="/cgi-bin/wa?LOGON=A3%3Dind1702%26L%3DES-TEX%26E%3Dquoted-printable%26P%3D599912%26B%3D--f403045f532cf07c0005495e868b%26T%3Dtext%252Fplain%3B%2520charset%3DUTF-8%26header%3D1" target="_parent" >[conectar para ver]</a>

\end{lstlisting}

\subsection{Día 25 19:00:55, Pablo Rodríguez}

\begin{lstlisting}
On 02/25/2017 05:59 PM, alberto alejandro moyano wrote:
> [...]
> Una pregunta que me hice en varias oportunidades, es porqué LaTeX no
> pudo/puede entrar en el mundo profesional de la edición, se que hay
> excepciones, pero sobran dedos de la mano para contar.
Alberto,
¿y cómo respondes a la pregunta de por qué TeX no entra en el mundo de
la edición profesional?
Pablo
-- 
<a href="http://www.ousia.tk" target="_blank">http://www.ousia.tk</a>
Si tiene algun problema con la utilizacion de la lista.
Pongase en contacto con nosotros a traves de:
<a href="/cgi-bin/wa?LOGON=A2%3Dind1702%26L%3DES-TEX%26F%3D%26S%3D%26P%3D44809" target="_parent" >[conectar para ver]</a>

\end{lstlisting}

\subsection{Día 25 22:27:37, alberto alejandro moyano}

\begin{lstlisting}
El 25 de febrero de 2017, 15:00, Pablo Rodríguez <<a href="/cgi-bin/wa?LOGON=A3%3Dind1702%26L%3DES-TEX%26E%3Dquoted-printable%26P%3D627452%26B%3D--94eb2c191e18fbb20605496183f9%26T%3Dtext%252Fplain%3B%2520charset%3DUTF-8%26header%3D1" target="_parent" >[conectar para ver]</a>> escribió:
> Alberto,
>
> ¿y cómo respondes a la pregunta de por qué TeX no entra en el mundo de
> la edición profesional?
> ...(texto omitido)...
Me limito a contar mi experiencia personal (con nombre y apellido), son
situaciones de hace algunos años, quiero decir, no son necesariamente las
únicas situaciones a existir al día de hoy.
Prometeo libros (http://www.prometeoeditorial.com)
Editorial de ciencias sociales, trabajé para ellos de manera externa
durante algunos años (les arme cerca de 30 libros), cuando se decidieron a
tener departamento de armado propio (interno) y les propuse LaTeX como
alternativa, su respuesta fue:
«si pongo un aviso clasificado pidiendo armadores en InDesign, obtengo
30 *curriculum
vitae* como mínimo, si lo hago con LaTeX, no viene nadie; no me puedo dar
el lujo de que una o pocas personas tengan el control de producción de mi
editorial».
Editorial Akadia (http://www.editorialakadia.com.ar)
Trabajé varios años (unos 50 libros) el dueño es un tipo raro --es todo un
precámbrico-- nunca ví a alguien tan alejado del mundo del libro, haciendo
libros (i.e. un tipo que no lee), pero es rápido con lo números, y un día
decidió tener taller propio de armado, su respuesta fue:
«con InDesign o Quark puedo manejar el costo salarial de los empleados, con
LaTeX, no».
Alfaomega (http://www.alfaomega.com.mx/default/argentina)
Editorial de libros técnicos orientados a ciencias exactas, trabajo para
ellos desde hace ya algunos años (a promedio de 1 libro de 900 páginas por
año), básicamente los libros que los autores entregan en LaTeX (en su
mayoría ingenieros), mi relación siempre fue con el gerente de producción,
pero un día me conoció el director general (mexicano) y me dijo:
«te tendría que haber conocido hace 4 años cuando nos vinimos a la
Argentina, ahora ya está todo armado sobre la plataforma de Adobe».
Edhasa (http://www.edhasa.com.ar)
Editorial de ciencias sociales, tuve 2 demos de producción, la conclusión
de la gerente fue:
«no puedo tener un sistema de producción que yo no manejo o me llevaría
mucho tiempo aprender, el curso de InDesign lo hice en un bimestre y
aprendí todo lo que necesito saber controlar».
Editorial La Ley (https://www.thomsonreuters.com.ar/es/tienda.html)
Editorial especializada en libros de leyes, mucha info, muchas notas a pie,
muchas referencias cruzadas, mucha bibliografía --música para los oídos de
LaTeX--. Tuve 2 reuniones con la gerente de producción, hice una demo,
quedo fascinada y su respuesta fue:
«... acabo de cerrar el acuerdo de 45 licencias de Adobe y Microsoft, no
puedo ir a gerencia a decir que ahora encontré algo mejor y más barato, al
margen de que, de los 70 empleados involucrados en producción probablemente
queden 40, lo lamento».
Revista Kairos (http://ar.kairosweb.com/k_revista.html)
Esta revista --decirle revista es un despropósito-- es un mamotreto de 600
páginas en tipografía cuerpo 8, es básicamente un nomenclador, indexan todo
en un motor sql (usan SQL Server) con una interfaz para el ABM y arman en
Corel Ventura 10, estuve reunido con ellos en 3 oportunidades, la respuesta
que obtuve de ellos fue:
«... es muy alto el costo de aprendizaje del personal afectado a la
producción --más exactamente con el servidor-- que debe cambiar todo el
*taggeado* que genera para Ventura, pasarlo a LaTeX».
Y la frutilla del postre.
Durante 2009/2010, dicte un seminario optativo de LaTeX (4 cuatrimestres,
cerca de 100 alumnos en total) en la Universidad de Buenos Aires, Facultad
de Filosofía y Letras, como anexo de la carrera de edición (
http://edicion.filo.uba.ar), el director de la carrera en ese momento es
amigo mío, la idea era que los estudiantes de la carrera de edición (más
concretamente en la materia «Producción editorial de libros científicos»)
tuvieran una mirada más amplia sobre las alternativas que existen. Cuando
mi amigo dejo su cargo, la nueva directora me dijo:
«... esto es realmente muy bueno, deja que lo analicemos y te llamamos»,
hoy estamos en 2017, a vos te llamaron..., a mí tampoco.
Saludos
Si tiene algun problema con la utilizacion de la lista.
Pongase en contacto con nosotros a traves de:
<a href="/cgi-bin/wa?LOGON=A3%3Dind1702%26L%3DES-TEX%26E%3Dquoted-printable%26P%3D627452%26B%3D--94eb2c191e18fbb20605496183f9%26T%3Dtext%252Fplain%3B%2520charset%3DUTF-8%26header%3D1" target="_parent" >[conectar para ver]</a>

\end{lstlisting}

\subsection{Día 26 06:37:24, Juan Manuel Macías}

\begin{lstlisting}
El panorama en España es igual de desolador. Por mi experiencia, si sacamos
un libro montado con TeX adelante, es desde "abajo", desde el autor, y no
desde la editorial. Salvo el tiempo que estuve maquetando libros para la
extinta y añorada DVD Ediciones, o la revista Cuaderno Ático, que dirijo,
son los departamentos, los autores o los proyectos de investigación los que
me encargan el montaje de los libros, y luego se llega a un acuerdo con
alguna editorial que no ponga reparos en que se le entregue la tripa del
libro terminada y cerrada, naturalmente dentro de ciertos márgenes de
estilo y formato que la propia editorial quiera marcar. Y aun así,
trabajando con Logos Verlag, Dikynson, Pórtico Libros y alguna más he
contado con bastante libertad. Precisamente acabamos de sacar en Pórtico el
Diccionario de personajes de la comedia antigua, íntegramente realizado en
TeX (salvo la cubierta). La gente de Pórtico quedó encantada con los
resultados que ofrece TeX y están muy interesados. En Dikynson también ha
salido hace unos meses la edición crítica, en formato bilingüe, de las
cuatro Filípicas de Demóstenes. Y ahora ando montando una edición crítica
de Caritón (la primera que hago en LuaLaTeX) para una editorial de
Heildelberg. En fin, son pequeños oasis en el desierto.
Lo que comentas, Alberto, me ha hecho recordar cuando la editorial Vaso
Roto, donde publiqué mi traducción de un libro de María Polydouri, me
propuso si quería maquetar algunos libros para ellos. Yo les dije que sí,
claro, pero que trabajaba en TeX. Además InDesign no lo uso desde 2007 por
mis reparos hacia usar software privativo. Cuando les expliqué un poco el
asunto, me dijeron que nones, porque ellos trabajaban con InDesign para
tener un control absoluto sobre las maquetaciones. Y una pequeña anécdota.
Cuando mi libro de Polydouri estaba a punto de entrar en imprenta, me
enviaron los ferros y la cubierta. Y casi me da un infarto. La editora, que
tiene InDesign instalado en su flamante Mac (pero se maneja justito en él)
decidió introducir por su cuenta unos cambios en la cubierta sin mi
consentimiento. Añadió un "Antología poética", cuando el libro *no* era una
antología poética. Y menos mal que lo pudimos remediar a tiempo. Y del
proceso de corrección de pruebas de mi traducción de la Poesía completa de
Cavafis en Pre-TeXtos ya ni hablo, porque eso fue un Via Crucis de meses,
Saludos,
Juan Manuel
El 25 de febrero de 2017, 22:27, alberto alejandro moyano <
<a href="/cgi-bin/wa?LOGON=A3%3Dind1702%26L%3DES-TEX%26E%3Dquoted-printable%26P%3D666298%26B%3D--001a114a50e08daf640549685b56%26T%3Dtext%252Fplain%3B%2520charset%3Dutf-8%26header%3D1" target="_parent" >[conectar para ver]</a>> escribió:
>
> El 25 de febrero de 2017, 15:00, Pablo Rodríguez <<a href="/cgi-bin/wa?LOGON=A3%3Dind1702%26L%3DES-TEX%26E%3Dquoted-printable%26P%3D666298%26B%3D--001a114a50e08daf640549685b56%26T%3Dtext%252Fplain%3B%2520charset%3Dutf-8%26header%3D1" target="_parent" >[conectar para ver]</a>> escribió:
>
>> Alberto,
> ...(texto omitido)...
Los artículos de ES-TEX son distribuidos gracias al apoyo y colaboración 
técnica de RedIRIS - Red Académica española - (http://www.rediris.es)

\end{lstlisting}

\subsection{Día 28 00:52:27, Aradenatorix Veckhôm Awecaelus}

\begin{lstlisting}
Pues es desolador ver la cerrazón y lo difícil que es poder trabajar
en el mundo editorial desde TeX. Yo acabo de ver una vacante en la
facultad de diseño perteneciente a la universidad en donde estudié y
en efecto, aunque se desea realizar documentos académicos donde TeX
queda como anillo al dedo, solicitan gente que sepa hacerlo en
InDesign.
Tal parece que para trabajar con TeX en el mundo editorial o te haces
de tu propia editorial o colaboras con alguna de las contadisimas
editoriales que han optado por usar alguna encarnación de TeX.
Saludos

\end{lstlisting}

\subsection{Día 28 22:03:55, Pablo Rodríguez}

\begin{lstlisting}
On 02/28/2017 12:52 AM, Aradenatorix Veckhôm Awecaelus wrote:
> Pues es desolador ver la cerrazón y lo difícil que es poder trabajar
> en el mundo editorial desde TeX. Yo acabo de ver una vacante en la
> facultad de diseño perteneciente a la universidad en donde estudié y
> en efecto, aunque se desea realizar documentos académicos donde TeX
> ...(texto omitido)...
InDesign es un estándar en un aspecto muy concreto. No de trabajo, no de
formato, no de rendimiento... es de expectativa.
Si te sales de ahí, el problema es que les rompes el saque. No se
esperan que pueda haber nada fuera de lo que esperan y conocen. Así la
gente se empieza a poner incómoda.
Quizá es que somos (todo el mundo) bastante gregarios. Lo lógico es que
no te juzguen por marcas, sino por rendimiento. Porque, ¿qué pasa si yo
escribo directamente el documento PDF como escribo este texto?
Obviamente es imposible lo anterior, pero la idea es fijarse en lo que
se logra y en cuánto tiempo. Lo demás carece de relevancia.
> Tal parece que para trabajar con TeX en el mundo editorial o te haces
> de tu propia editorial o colaboras con alguna de las contadisimas
> editoriales que han optado por usar alguna encarnación de TeX.
No lo sé, si es una editorial pequeña, lo que querrán es rapidez y
buenos resultados. O no les importará en qué hagas las cosas, siempre
que las hagas rápido.
Pero es cierto que tiene que ser una editorial muy pequeña.
Saludos,
Pablo
-- 
<a href="http://www.ousia.tk" target="_blank">http://www.ousia.tk</a>

\end{lstlisting}
\section{Listas enumeradas por sección}

\subsection{Día 21 01:56:54, Aradenatorix Veckhôm Awecaelus}

\begin{lstlisting}
Hola:
Tengo que realizar documentos en mi trabajo de acuerdo al método
geométrico de Baruch Spinoza por lo cual, cualquier idea se desarrolla
en forma de listas numeradas y anidadas (a veces con más de 6
niveles), lo cual no es difícil de realizar en LaTeX.
Partiendo del hecho de que dichos documentos se realizan con las
clases article o scrartcl, mi duda es ¿cómo hacer que el número
inicial de la lista coincida con el de la sección en cuestión?
Por ejemplo si tengo:
\section{Título de mi sección}
\begin{enumerate}
\item Primer punto de mi sección
\begin{enumerate}
\item Primer subpunto de mi sección
\item Segundo subpunto de mi sección
\end{enumerate}
\end{enumerate}
¿Hay forma de hacer que si compilo esto y es la sección 3, aparezca como:
3. Título de mi sección
3.1. Primer punto de mi sección
3.1.1. Primer subpunto de mi sección
3.1.2. Segundo subpunto de mi sección
Asumo que tendré que agregar algo al preámbulo y casi seguro que tiene
que ver con el paquete enumitem.
Saludos
Los artículos de ES-TEX son distribuidos gracias al apoyo y colaboración 
técnica de RedIRIS - Red Académica española - (<a href="http://www.rediris.es" target="_blank">http://www.rediris.es</a>)

\end{lstlisting}

\subsection{Día 21 09:43:21, Ignasi}

\begin{lstlisting}
¿Necesitas secciones? ¿No pueden ser el primer nivel de la enumeración?
Ignasi
El 21/02/2017 a les 1:56, Aradenatorix Veckhôm Awecaelus ha escrit:
> Hola:
>
> Tengo que realizar documentos en mi trabajo de acuerdo al método
> geométrico de Baruch Spinoza por lo cual, cualquier idea se desarrolla
> ...(texto omitido)...
Aquest correu electrònic s'ha verificat mitjançant l'Avast antivirus.
<a href="https://www.avast.com/antivirus" target="_blank">https://www.avast.com/antivirus</a>
Los artículos de ES-TEX son distribuidos gracias al apoyo y colaboración 
técnica de RedIRIS - Red Académica española - (<a href="http://www.rediris.es" target="_blank">http://www.rediris.es</a>)

\end{lstlisting}

\subsection{Día 21 09:47:05, Aradenatorix Veckhôm Awecaelus}

\begin{lstlisting}
Necesito secciones y en ocasiones subsecciones, de otro modo con
Enumitem lo tendría resuelto, y es justamente para facilitar la
generación de la tabla de contenidos.
Gracias
Los artículos de ES-TEX son distribuidos gracias al apoyo y colaboración 
técnica de RedIRIS - Red Académica española - (<a href="http://www.rediris.es" target="_blank">http://www.rediris.es</a>)

\end{lstlisting}

\subsection{Día 21 14:14:09, Juan Luis Varona Malumbres}

\begin{lstlisting}
A ver si esto te vale:
\usepackage{enumitem}
\begin{enumerate}[label=\thesection.\arabic*.]
\item Hola
\begin{enumerate}[label=\theenumi\arabic*.]
\item Hola
\item Hola
\begin{enumerate}[label=\theenumii\arabic*.]
\item Hola
\item Hola
\item Hola
\end{enumerate}
\item Hola
\end{enumerate}
\item Hola
\item Hola
\end{enumerate}
Saludos,
Juan Luis
El 21/02/2017, a las 09:47 , Aradenatorix Veckh�m Awecaelus <<a href="/cgi-bin/wa?LOGON=A2%3Dind1702%26L%3DES-TEX%26F%3D%26S%3D%26P%3D28062" target="_parent" >[conectar para ver]</a>> escribi�:
> Necesito secciones y en ocasiones subsecciones, de otro modo con
> Enumitem lo tendr�a resuelto, y es justamente para facilitar la
> generaci�n de la tabla de contenidos.
> 
> ...(texto omitido)...
Los art�culos de ES-TEX son distribuidos gracias al apoyo y colaboraci�n 
t�cnica de RedIRIS - Red Acad�mica espa�ola - (<a href="http://www.rediris.es" target="_blank">http://www.rediris.es</a>)

\end{lstlisting}

\subsection{Día 22 08:06:51, Aradenatorix Veckhôm Awecaelus}

\begin{lstlisting}
Hola Juan Luis
Está perfecto el ejemplo que has enviado. Ahora mi duda es si con
enumitem se puede "automatizar" esto desde el preámbulo mismo para que
cada vez que use el entorno enumerate se generen las listas de esta
forma sin que tenga que poner la opción cada vez que anide un nuevo
subnivel en la lista.
Necesito ampliar la cantidad máxima de subniveles permitidos, ¿suegerncias?
Archivos de ES-TEX: <a href="http://listserv.rediris.es/archives/es-tex.html" target="_blank">http://listserv.rediris.es/archives/es-tex.html</a>

\end{lstlisting}

\subsection{Día 22 11:41:02, Juan José Torrens}

\begin{lstlisting}
El 22/2/17 a las 8:06, Aradenatorix Veckhôm Awecaelus escribió:
> Hola Juan Luis
>
>
> Está perfecto el ejemplo que has enviado. Ahora mi duda es si con
> ...(texto omitido)...
Para aumentar el número de niveles, necesitas «clonar» el entorno 
enumerate, de modo que puedas previamente cambiar el número máximo de 
niveles y generar así los contadores necesarios. Lo tienes todo en la 
documentación de enumitem (sección 7): 
<a href="http://texdoc.net/texmf-dist/doc/latex/enumitem/enumitem.pdf" target="_blank">http://texdoc.net/texmf-dist/doc/latex/enumitem/enumitem.pdf</a>
He preparado un ejemplo completo, que puedes ver en Overleaf: 
<a href="https://www.overleaf.com/read/nxryprkmtjkr" target="_blank">https://www.overleaf.com/read/nxryprkmtjkr</a>
He puesto ocho niveles, número que puedes cambiar fácilmente. Tendrás 
que adaptar las órdenes \setlist de cada nivel a tus necesidades.
Un saludo
Juanjo
Archivos de ES-TEX: <a href="http://listserv.rediris.es/archives/es-tex.html" target="_blank">http://listserv.rediris.es/archives/es-tex.html</a>

\end{lstlisting}

\subsection{Día 22 23:53:34, Aradenatorix Veckhôm Awecaelus}

\begin{lstlisting}
Muchas gracias Juanjo, lo reviso.
Saludos
Archivos de ES-TEX: <a href="http://listserv.rediris.es/archives/es-tex.html" target="_blank">http://listserv.rediris.es/archives/es-tex.html</a>

\end{lstlisting}
\section{Imprenta Sur (off topic)}

\subsection{Día 22 01:18:42, Juan Manuel Macías}

\begin{lstlisting}
Hola a todos. Hace unos años, con motivo de un encuentro de traductores de
griego en Málaga, tuve ocasión de visitar la mítica Imprenta Sur, invitado
amablemente por el Centro Cultural de la Generación del 27. Para quien no
le suene, ésta era la pequeña imprenta (apenas una habitación) donde Prados
y Altolaguirre imprimían la famosa revista Litoral. Si os apetece echar un
ojo, he subido a mi servidor de Nextcloud unas cuantas fotos que hice,
bastante malas:
http://www.revistacuadernoatico.com/nextcloud/index.php/s/YPiQYatmYlWzCcx
Lo curioso es que esta imprenta sigue hoy día operativa, a cargo de una
única persona, y saca de vez en cuando alguna que otra publicación
especial, relacionada con el 27. El teléfono que aparece sobre los
chibaletes (las cajoneras que guardan los tipos en plomo) también funciona.
En las dos últimas fotos aparecen las Minervas, las máquinas encargadas de
imprimir la página. Es una pena que en aquella visita la persona que lleva
la imprenta estuviese de baja por enfermedad, porque habría sido un placer
hablar con ella.
Siempre he pensado que TeX es el heredero natural y "legítimo", en el mundo
binario, de estas imprentas...
Saludos,
JM
Archivos de ES-TEX: http://listserv.rediris.es/archives/es-tex.html

\end{lstlisting}

\subsection{Día 22 03:34:21, Aradenatorix Veckhôm Awecaelus}

\begin{lstlisting}
Gracias Juan Manuel por este aporte, lamentablemente  el enlace de
nextcloud no está disponible, arroja el siguiente mensaje de error:
Internal Server Error
The server encountered an internal error or misconfiguration and was
unable to complete your request.
Please contact the server administrator, <a href="/cgi-bin/wa?LOGON=A2%3Dind1702%26L%3DES-TEX%26F%3D%26S%3D%26P%3D30069" target="_parent" >[conectar para ver]</a> and inform
them of the time the error occurred, and anything you might have done
that may have caused the error.
More information about this error may be available in the server error log.
Me parece interesante el trabajo de esa imprenta y que siga vigente
hasta nuestros días.
Saludos
Archivos de ES-TEX: <a href="http://listserv.rediris.es/archives/es-tex.html" target="_blank">http://listserv.rediris.es/archives/es-tex.html</a>

\end{lstlisting}

\subsection{Día 22 03:50:19, Juan Manuel Macías}

\begin{lstlisting}
Mil disculpas. Se cayó un rato el servidor, pero creo que ya funciona 
bien. Por si acaso, dejo este otro enlace de dropbox: 
<a href="https://www.dropbox.com/sh/99no30vl8tk64jd/AAAtEuLAIcwrLnUFvF568j-Ca?dl=0" target="_blank">https://www.dropbox.com/sh/99no30vl8tk64jd/AAAtEuLAIcwrLnUFvF568j-Ca?dl=0</a>
Las fotos no son gran cosa, pero si vais a Málaga, merece la pena hacer 
una visita a la imprenta.
Saludos,
JM
El 22/02/17 a las 03:34, Aradenatorix Veckhôm Awecaelus escribió:
> Gracias Juan Manuel por este aporte, lamentablemente  el enlace de
> nextcloud no está disponible, arroja el siguiente mensaje de error:
>
> Internal Server Error
> ...(texto omitido)...
Archivos de ES-TEX: <a href="http://listserv.rediris.es/archives/es-tex.html" target="_blank">http://listserv.rediris.es/archives/es-tex.html</a>

\end{lstlisting}

\subsection{Día 22 07:29:01, Aradenatorix Veckhôm Awecaelus}

\begin{lstlisting}
Gracias, ya se ve en Nextcloud.
Saludos
Archivos de ES-TEX: <a href="http://listserv.rediris.es/archives/es-tex.html" target="_blank">http://listserv.rediris.es/archives/es-tex.html</a>

\end{lstlisting}
\section{headsep}

\subsection{Día 22 19:49:53, Santiago Higuera}

\begin{lstlisting}
Hola:
A ver si me podéis ayudar.
Cuando pongo los siguientes comandos, no tengo separación entre la línea 
bajo la cabecera de página y el texto del cuerpo de la página. Es como 
si no hiciera caso del comando 'headsep'. No sé si es que colisionan los 
paquetes fancyhdr y geometry, o algo parecido.
% --------------------------------------------------------------
% geometry: formato de páginas
% --------------------------------------------------------------
%\usepackage{showframe}
\usepackage{geometry}
\geometry{
  a4paper,
  top=50mm,
  headheight=20mm,
  headsep=15mm,
  bottom=20mm,
  inner=30mm,
  outer=25mm
}
% --------------------------------------------------------------
% --------------------------------------------------------------
% fancypage: encabezados
% --------------------------------------------------------------
\usepackage{fancyhdr}
\fancyhf{}%
\fancyhead[LO]{\fontsize{10}{12}\selectfont\nouppercase\leftmark}
\fancyhead[RE]{S. Higuera de Frutos}
\fancyfoot[C]{\thepage}
\renewcommand{\headrulewidth}{1pt}
% Redefine the plain page style applied to chapter first page
\fancypagestyle{plain}{%
   \fancyhf{}%
   \renewcommand{\headrulewidth}{0pt}% Line at the header invisible
   %\renewcommand{\footrulewidth}{0.4pt}% Line at the footer visible
}
% ----------------
Gracias anticipadas
Un saludo
Santiago Higuera
Archivos de ES-TEX: <a href="http://listserv.rediris.es/archives/es-tex.html" target="_blank">http://listserv.rediris.es/archives/es-tex.html</a>

\end{lstlisting}

\subsection{Día 23 10:17:39, Santiago Higuera}

\begin{lstlisting}
Bueno, creo que lo he resuelto, añadiendo explícitamente el parámetro 
'textheight' dentro del comando \geometry, aunque la verdad es que no 
termino de entender como suma unos parámetros con otros.
El 22/02/17 a las 19:49, Santiago Higuera escribió:
> Hola:
>
> A ver si me podéis ayudar.
>
> ...(texto omitido)...

\end{lstlisting}
\section{puntuación en folios de páginas superiores a 999}

\subsection{Día 23 19:04:15, alberto alejandro moyano}

\begin{lstlisting}
Hola colisteros
Estoy dando vueltas, investigando en toda la biblio que tengo en casa, más
internet y no consigo respuesta; la pregunta es simple ¿se coloca punto
divisor de unidad de mil en los folios de páginas mayores a 999?
Hice un ejercicio en San LaTeX (simule un libro de 1400 páginas) para ver
como lo hace, y veo que no pone punto divisor de unidad de mil, pero
tampoco sé si esto obedece a la tradición anglosajona (como con el tema de
foliado romano, ellos usan minúsculas y en español se usa mayúscula).
El problema que trato tiene que ver con referencias bibliográficas con
marcas del tipo 1121-1124.
En algún lugar, hace ya no sé cuanto recuerdo haber leído que solamente los
años no llevan divisor de unidad de mil (por ejemplo el año 2017), de ahí
la duda.
Cualquier rayo de luz, se agradece.
Alberto
======================

\end{lstlisting}

\subsection{Día 23 19:16:19, Rubén Gómez Antolí}

\begin{lstlisting}
El 23/02/17 a las 19:04, alberto alejandro moyano escribió:
> Hola colisteros
> 
> Estoy dando vueltas, investigando en toda la biblio que tengo en casa,
> más internet y no consigo respuesta; la pregunta es simple ¿se coloca
> ...(texto omitido)...
Según la ortografía de la RAE (y copio tal cual pone):
«b) en los números que indican paginación o numeración de versos,
columnas, etc.: página 14881, verso 1756. De forma análoga al caso
anterior, sí podrá aplicarse el espaciado si el número expresa cantidad:
el sumario judicial tiene más de 12 500 páginas.»
Página 664 de «Ortografía de la lengua española», edición de 2010.
Espero que sirva.
Salud y Revolución.
Lobo.
-- 
Libertad es poder elegir en cualquier momento. Ahora yo elijo GNU/Linux,
para no atar mis manos con las cadenas del soft propietario.
Porque la libertad no es tu derecho, es tu responsabilidad.
<a href="http://www.mucharuina.com" target="_blank">http://www.mucharuina.com</a>
Desde El Ejido, en Almería, usuario registrado Linux #294013
<a href="http://www.counter.li.org" target="_blank">http://www.counter.li.org</a>

\end{lstlisting}
\section{Alineamiento óptico de márgenes en LuaTeX / XeTeX}

\subsection{Día 24 11:07:39, Juan Manuel Macías}

\begin{lstlisting}
Hola a todos,
Llevo unos días devanándome con la posibilidad de aplicar en LuaLaTeX o 
XeLaTeX un par de etiquetas Open Type un tanto exóticas: lfbd y rtbd 
(left bounds /right bounds), que pertenecen a la lista de tablas de 
posición (GPOS) y se encargarían de alinear los caracteres o marcas 
diacríticas afectados fuera de los límites izquierdo y derecho del 
texto, respectivamente. Y sólo en esos contextos, pues en el interior de 
la línea no tendrían efecto. Vendría a ser, supongo, la versión opentype 
de lo que el paquete microtype denomina «protrusion». El caso es que la 
información que he podido ir recavando en la red es poca y ambigua, 
sumado a que casi ninguna fuente que conozco tiene incorporadas estas 
características en sus tablas open type. La única excepción la he visto 
en las fuentes Linux Libertine, donde sí aparecen, para los glifos V y 
W, pero de una forma que me desconcierta. Determina open type que ambas 
etiquetas deben ser llamadas por una tercera, intermediaria, opbd 
(optical bounds). Al abrir con Fontforge los lookups de la Linux 
Libertine, aparecen, sí, marcados unos valores de desplazamiento para 
las dos etiquetas lfbd y rtbd, pero ni sombra de la opbd... En fin, he 
editado otra fuente añadiendo algunas tablas lfbd y rtbd a un par de 
letras griegas, con unos valores algo exagerados (para ver claramente el 
efecto), y he probado a cargarla en LuaLaTeX, tanto mediante fontspec 
(activando las etiquetas con RawFeature=) como directamente con 
luaotfload. Activando la etiqueta «fantasma» +opbd no sucede nada, como 
me temía. Activando las dos etiquetas +lfbd / +rtbd, sí se produce el 
desplazamiento del glifo, pero no sólo en los márgenes izquierdo y 
derecho sino también en interior de línea, lo cual es un funcionamiento 
inadmisible. Y aquí me he quedado. Me pregunto si LuaTeX soportará 
realmente esta característica open type. Y, en cualquier caso, el hecho 
de que los efectos hayan de darse justo en ambos extremos de la línea, 
¿no será una tarea que deba caer de la parte del programa que genera el 
texto, y es el encargado de decidir cuándo cortar las líneas, más que de 
Open Type y de la fuente de turno? ¿Puede ser tal vez un mal 
planteamiento de Open Type estas características de márgenes ópticos?
Disculpad de antemano que abra este tema tan marginal (nunca mejor 
dicho), pero aquí lo dejo por si a alguien le puede interesar o ha 
avanzado más que yo con estos tráfagos.
Saludos,
Juan Manuel
Para darse de baja ES-TEX pincha y envia el siguiente url
mailto:[conectar para ver]

\end{lstlisting}

\subsection{Día 25 16:15:12, Pablo Rodríguez}

\begin{lstlisting}
On 02/24/2017 11:07 AM, Juan Manuel Macías wrote:
> [...]
> En fin, he editado otra fuente añadiendo algunas tablas lfbd y rtbd a
> un par de letras griegas [...] activando las dos etiquetas +lfbd /
> +rtbd, sí se produce el [...] también en interior de línea, lo cual es un
> ...(texto omitido)...
Hola Juan Manuel,
no sé si sería posible que pusieses un enlace con la tipografía
modificada. Sería para probarlo con ConTeXt y mandar un fallo de error a
los desarrolladores en caso de que no funcionase correctamente.
> Y, en cualquier caso, el hecho 
> de que los efectos hayan de darse justo en ambos extremos de la línea, 
> ¿no será una tarea que deba caer de la parte del programa que genera el 
> texto, y es el encargado de decidir cuándo cortar las líneas, más que de 
> ...(texto omitido)...
Respecto a tu pregunta, creo que no. Porque entiendo que es una tarea
compartida entre programa y el glifo concreto.
Partir la línea es una cuestión del programa, seguro. Pero la
información de cuánto debe sobresalir por ambos lados, debe depender del
glifo.
Tomo el ejemplo de las marcas diacríticas de letras mayúsuclas en griego
politónico. Marcar un porcentaje (como sucede con la prominencia en la
línea de ciertos caracteres) es fácil que falle. Porque ¿cómo sabemos en
qué proporción tiene que salir el espíritu (suave o áspero) de la
épsilon mayúscula respecto al glifo concreto?
Al tener que considerar esa información para partir la línea, la
cuestión está que sólo lo considere en los casos marginales.
Saludos,
Pablo
-- 
<a href="http://www.ousia.tk" target="_blank">http://www.ousia.tk</a>
Si tiene algun problema con la utilizacion de la lista.
Pongase en contacto con nosotros a traves de:
<a href="/cgi-bin/wa?LOGON=A2%3Dind1702%26L%3DES-TEX%26F%3D%26S%3D%26P%3D42431" target="_parent" >[conectar para ver]</a>

\end{lstlisting}

\subsection{Día 26 06:01:19, Juan Manuel Macías}

\begin{lstlisting}
Hola, Pablo,
Muchas gracias por tu interés. El caso es que, como comentaba, no he 
sido capaz de encontrar apenas información sobre esto en ninguna de las 
partes (Opentype, LuaTeX,XeTeX, etc). Y a eso se une la dificultad de 
encontrar un buen especimen de fuente que incorpore claramente la 
característica. He mirado en fuentes y más fuentes de todo pelaje y la 
única que parece incorporarla es la Linux Libertine. Puedes encontrar 
fácilmente las etiquetas lfbd y rtbd si abres la fuente en Fontforge y 
accedes a Elemento -> Atributos fuente -> Lookups -> GPOS. Pero si abres 
los datos de cada etiqueta sólo encontrarás parámetros de 
desplazamiento. Por ejemplo, para lfbd en la letra V, Δx=-27 (borde 
izquierdo de la letra), Δx_adv=-27 (borde derecho). Y si creas una nueva 
etiqueta lfbd (add Lookup/Tipo: Single position/lfbd), sólo te da a 
elegir entre los cuatro parámetros de desplazamiento horizontal y 
vertical: Δx, Δx_adv, Δy, Δy_adv. Y si creas la etiqueta opbd (optical 
bounds) es exactamente el mismo procedimiento. ¿Con esos simples valores 
realmente sabe la fuente si el glifo está en el margen izquierdo o 
derecho del texto? Creo que no, o algo falta, o opentype lo ha dejado in 
media re, y de ahí mi pregunta final en el otro mensaje. Por ejemplo, si 
yo quiero añadir una etiqueta de sustitución contextual (GSUB), como la 
de la beta curvada sin descendente en mitad de palabra, primero he de 
introducir la sustitución de glifos y acto seguido (porque con ese 
simple dato la fuente sustituiría la letra en todos los contextos), he 
de declarar una cadena condicional de caracteres, para que ahí, y sólo 
ahí, se produzca el cambio. Naturalmente, si cargo en LuaLaTeX o XeLaTeX 
las etiquetas lfbd y rtbd, el desplazamiento de la letra se produce 
siempre, pero no hay forma de controlar la posición contextual de los 
márgenes.
No sé si por el lado de ConTeXt habrá habido algún avance. Buscando 
información, encontré este hilo de 2010 donde Khaled Hosny y Hans Hagen 
discutían el asunto y la posibilidad de empezar a hacer pruebas: 
https://mailman.ntg.nl/pipermail/dev-context/2010/002139.html
Saludos,
Juan Manuel
El 25/02/17 a las 16:15, Pablo Rodríguez escribió:
> On 02/24/2017 11:07 AM, Juan Manuel Macías wrote:
>> [...]
>> En fin, he editado otra fuente añadiendo algunas tablas lfbd y rtbd a
>> un par de letras griegas [...] activando las dos etiquetas +lfbd /
> ...(texto omitido)...
-- 
*Juan Manuel Macías*
http://diosas-nubes.blogspot.com <http://diosas-nubes.blogspot.com/>
http://www.revistacuadernoatico.com <http://www.revistacuadernoatico.com/>
Los artículos de ES-TEX son distribuidos gracias al apoyo y colaboración 
técnica de RedIRIS - Red Académica española - (http://www.rediris.es)

\end{lstlisting}

\subsection{Día 26 10:19:29, Pablo Rodríguez}

\begin{lstlisting}
On 02/26/2017 06:01 AM, Juan Manuel Macías wrote:
> Hola, Pablo,
> 
> Muchas gracias por tu interés. El caso es que, como comentaba, no he
> sido capaz de encontrar apenas información sobre esto en ninguna de las
> ...(texto omitido)...
Juan Manuel,
lo de Linux Libertine y los márgenes (voy a traducir así “bounds”)
ópticos parece más una prueba de funcionamiento que otra cosa.
De las lenguas que conozco, esa propiedad sólo tiene sentido en las
marcas diacríticas de las mayúsculas griegas. Como sustitución de la
prominencia en la línea (la “protrusion”), no creo que deba usarse.
> Puedes encontrar fácilmente las etiquetas lfbd y rtbd si abres la
> fuente en Fontforge y accedes a Elemento -> Atributos fuente ->
> Lookups -> GPOS. Pero si abres los datos de cada etiqueta sólo
> encontrarás parámetros de desplazamiento. Por ejemplo, para lfbd en
> ...(texto omitido)...
Gracias por la aclaración. Lo hice en su momento, pero no tenía idea de
cómo. Sin duda, me servirá en el futuro.
> ¿Con esos simples valores realmente sabe la fuente si el glifo está
> en el margen izquierdo o derecho del texto? Creo que no, o algo
> falta, o opentype lo ha dejado in media re, y de ahí mi pregunta
> final en el otro mensaje.
No, el tipo de letra no sabe ni tiene que saber nada. Es el programa el
que la coloca a principio o final del línea y ahí es donde tiene que
leer el valor del margen óptico concreto.
> Por ejemplo, si yo quiero añadir una etiqueta de sustitución
> contextual (GSUB), como la de la beta curvada sin descendente en
> mitad de palabra, primero he de introducir la sustitución de glifos y
> acto seguido (porque con ese simple dato la fuente sustituiría la
> ...(texto omitido)...
Creo que la comparación es más compleja. No es tanto que sea tipo GPOS o
GSUB, como que el tipo de etiqueta requiera dos variables o una sola.
Por el tipo de etiqueta, lfbd o rtbd ya dan el contexto al glifo: cuando
estén situados al comienzo o al final de la línea.
liga sólo hace sustitución del glifo fi cuando después de f se da la i.
La segunda, entiendo, sería binaria (necesita dos variables), y las
primeras son unarias (necesitan una sola variable). Repito, porque la
propia etiqueta da el contexto, la condición necesaria para que se aplique.
> Naturalmente, si cargo en LuaLaTeX o XeLaTeX las etiquetas lfbd y
> rtbd, el desplazamiento de la letra se produce siempre, pero no hay
> forma de controlar la posición contextual de los márgenes.
Si lo aplica a todos, LuaLaTeX lo hace mal, porque sólo debe considerar
el valor al principio o al final de la línea.
La especificación no está mal definida (o es incompleta). Es el programa
quien lo hace mal.
> No sé si por el lado de ConTeXt habrá habido algún avance. Buscando
> información, encontré este hilo de 2010 donde Khaled Hosny y Hans Hagen
> discutían el asunto y la posibilidad de empezar a hacer pruebas:
> <a href="https://mailman.ntg.nl/pipermail/dev-context/2010/002139.html" target="_blank">https://mailman.ntg.nl/pipermail/dev-context/2010/002139.html</a>
En ConTeXt la cosa se quedó en su momento a medias (porque nadie mostró
más interés, supongo).
He preguntado en la lista de ConTeXt y Hans Hagen me ha dicho que
retomará el tema
(<a href="https://mailman.ntg.nl/pipermail/ntg-context/2017/088149.html)" target="_blank">https://mailman.ntg.nl/pipermail/ntg-context/2017/088149.html)</a>. En
cuanto esté arreglado, te aviso.
Por lo que entiendo del mensaje, LuaTeX (al menos la versión 1.0.3) está
preparado para opbd (así debe invocarse) y tiene que hacerse junto con
la prominencia en la línea.
Cómo haya de gestionarse eso con LaTeX, entre los paquetes microtype y
fontspec, pues sinceramente ya no lo sé.
<a href="http://mirrors.ctan.org/macros/latex/contrib/fontspec/fontspec.pdf#page=38" target="_blank">http://mirrors.ctan.org/macros/latex/contrib/fontspec/fontspec.pdf#page=38</a>
afirma que no está incorporada la propiedad de márgenes ópticos.
Creo que mejor esperar a que en ConTeXt funcione y luego ya intentaría
que lo incorporase fontspec. Sobre todo, porque sin un ejemplo será
todavía más complicado.
En cuanto sepa algo, te digo. Saludos,
Pablo
-- 
<a href="http://www.ousia.tk" target="_blank">http://www.ousia.tk</a>
Los artículos de ES-TEX son distribuidos gracias al apoyo y colaboración 
técnica de RedIRIS - Red Académica española - (<a href="http://www.rediris.es" target="_blank">http://www.rediris.es</a>)

\end{lstlisting}

\subsection{Día 26 11:02:33, Juan Manuel Macías}

\begin{lstlisting}
Hola, Pablo. Muchas gracias por todo.
> De las lenguas que conozco, esa propiedad sólo tiene sentido en las
> marcas diacríticas de las mayúsculas griegas. Como sustitución de la
> prominencia en la línea (la “protrusion”), no creo que deba usarse.
Cierto, y sólo en poesía, si hacemos caso a lo que decía Haralambous en
su famoso tratadillo. Habría que ver casos concretos, pues tengo aquí
delante unas fotos de una primera edición de 1927 de las Elegías y
sátiras de Karyotakis (en griego politónico) y no veo que uso ese
recurso. Habría que ver también si algún otro sistema de escritura
podría sacar ventaja de la característica...
> Por el tipo de etiqueta, lfbd o rtbd ya dan el contexto al glifo: cuando
> estén situados al comienzo o al final de la línea.
Sí, eso mismo sospechaba yo, que todo ya estaba incluido en la propia
etiqueta. El problema es que no existe ningún software donde
comprobarlo, al margen de que LuaTeX, teóricamente, pueda dar soporte a
la característica, así que mis dudas venían por no saber a quién echar
la culpa, si a Opentype o a LuaTeX. En efecto, ambas etiquetas han de
llamarse con opbd, pero los resultados son los que hemos venido
comentando hasta ahora.
> He preguntado en la lista de ConTeXt y Hans Hagen me ha dicho que
> retomará el tema
> (https://mailman.ntg.nl/pipermail/ntg-context/2017/088149.html). En
> cuanto esté arreglado, te aviso.
> ...(texto omitido)...
Interesantísimo todo, mil gracias. LuaLaTeX y Fontspec usan un port de
luaotfload de ConTeXt, tal vez se puedan hacer algunas pruebas en el
futuro con una fuente griega modificada. El paquete Microtype no tendría
que ver nada aquí, aunque use también "protrusion" en su jerga. En
teoría, con microtype se pueden asignar valores  de desplazamiento en
márgenes a letras concretas. Eso no lo he probado, otra cosa pendiente.
En fin, esperemos acontecimientos, como dices. Tengo que empezar a
echarle un vistazo a ConTeXt, pero como me meta en otro berenjenal no sé
cómo voy a acabar :-)
Saludos,
Juan Manuel
El 26/02/17 a las 10:19, Pablo Rodríguez escribió:
> On 02/26/2017 06:01 AM, Juan Manuel Macías wrote:
>> Hola, Pablo,
>>
>> Muchas gracias por tu interés. El caso es que, como comentaba, no he
> ...(texto omitido)...
-- 
*Juan Manuel Macías*
http://diosas-nubes.blogspot.com <http://diosas-nubes.blogspot.com/>
https://apuntestipograficos.wordpress.com
<https://apuntestipograficos.wordpress.com/>
http://www.revistacuadernoatico.com <http://www.revistacuadernoatico.com/>
Los artículos de ES-TEX son distribuidos gracias al apoyo y colaboración 
técnica de RedIRIS - Red Académica española - (http://www.rediris.es)

\end{lstlisting}

\subsection{Día 26 11:27:51, Juan Manuel Macías}

\begin{lstlisting}
Addendum (se me olvidaba), comentabas que Fontspec, en efecto, no da 
soporte (de momento) a la característica de los márgenes ópticos, pero 
se puede cargar igualmente con \defaultfontfeatures{RawFeature={+lfbd}}, 
que produce el mismo efecto que si se invoca mediante luaotfload: 
\font\linuxlib =file:libermac:;+opbd at 12pt
Saludos,
Juan Manuel
El 26/02/17 a las 11:02, Juan Manuel Macías escribió:
>
> Hola, Pablo. Muchas gracias por todo.
>
>> De las lenguas que conozco, esa propiedad sólo tiene sentido en las
> ...(texto omitido)...
-- 
*Juan Manuel Macías*
http://diosas-nubes.blogspot.com <http://diosas-nubes.blogspot.com/>
https://apuntestipograficos.wordpress.com 
<https://apuntestipograficos.wordpress.com/>
http://www.revistacuadernoatico.com <http://www.revistacuadernoatico.com/>
Los artículos de ES-TEX son distribuidos gracias al apoyo y colaboración 
técnica de RedIRIS - Red Académica española - (http://www.rediris.es)

\end{lstlisting}

\subsection{Día 26 12:16:51, Pablo Rodríguez}

\begin{lstlisting}
On 02/26/2017 11:02 AM, Juan Manuel Macías wrote:
> Hola, Pablo. Muchas gracias por todo.
De nada, Juan Manuel.
Un placer poder ayudarte para poder comtemplar también los resultados de
tus composiciones tipográficas.
Por cierto, la última entrada de tu bitácora: ¿es para una edición de
una traducción al griego antiguo (o moderno, pero politónico) de «Don
Quijote» con notas en español?
>> De las lenguas que conozco, esa propiedad sólo tiene sentido en las
>> marcas diacríticas de las mayúsculas griegas. Como sustitución de la
>> prominencia en la línea (la “protrusion”), no creo que deba usarse.
> 
> ...(texto omitido)...
De la vez que lo había consultado, me había quedado la idea de que eran
escrituras orientales que usaban ese sistema.
Pero por lo que parece, mi memoria me traiciona, porque los ejemplos son
otros totalmente distintos (aplicados a escrituras latinas):
<a href="https://www.microsoft.com/typography/otspec/features_ko.htm#lfbd" target="_blank">https://www.microsoft.com/typography/otspec/features_ko.htm#lfbd</a>
<a href="https://www.microsoft.com/typography/otspec/features_pt.htm#rtbd" target="_blank">https://www.microsoft.com/typography/otspec/features_pt.htm#rtbd</a>
Para eso, estoy de acuerdo con las conclusiones finales de este hilo:
<a href="http://www.typophile.com/node/100183" target="_blank">http://www.typophile.com/node/100183</a>. Como sustituto de la
microtipografía (de prominencia en la línea [y subsidiariamente, de la
elasticidad horizontal de los caracteres), no creo que sea necesario.
Entiendo que tiene sentido con glifos concretos, esto es, caracteres de
tipos de letra concretos. Así no hay que definirlos para cada
aplicación, con hacerlo en el archivo .ttf u .otf está ya para todo (y
para todos).
Respecto al ejemplo concreto de las marcas diacríticas de las mayúsculas
griegas, quizá también lo usaría en prosa. Al menos probaría para ver
cómo resulta.
> Interesantísimo todo, mil gracias. LuaLaTeX y Fontspec usan un port de
> luaotfload de ConTeXt, tal vez se puedan hacer algunas pruebas en el
> futuro con una fuente griega modificada.
En realidad, con Linux Libertine O se ve. Se ve muy poco, pero se ve.
Cuando funcione en ConTeXt, yo me encargo de que funcione (me pongo en
contacto con el autor) en fontspec.
> El paquete Microtype no tendría que ver nada aquí, aunque use también
> "protrusion" en su jerga. En teoría, con microtype se pueden asignar
> valores  de desplazamiento en márgenes a letras concretas. Eso no lo
> he probado, otra cosa pendiente. En fin, esperemos acontecimientos,
> ...(texto omitido)...
Haciéndolo con microtype el problema es que sólo te funciona usándolo
con microtype. Si lo tienes en el archivo de la tipografía digital, lo
tienes siempre que lo uses. Creo que es mejor así.
En la letra digital, lo tienes para todo lo que entienda los márgenes
ópticos. Si incorporas los valores de prominencia marginal en el
programa, te limitas a ese programa.
> Tengo que empezar a echarle un vistazo a ConTeXt, pero como me meta
> en otro berenjenal no sé cómo voy a acabar :-)
No tengas prisa. Hay tiempo para todo. Porque antes de que pruebes, te
pediría que esperases, porque hay algunas cosas que todavía hay que
pulir en ConTeXt.
Saludos,
Pablo
-- 
<a href="http://www.ousia.tk" target="_blank">http://www.ousia.tk</a>
Los artículos de ES-TEX son distribuidos gracias al apoyo y colaboración 
técnica de RedIRIS - Red Académica española - (<a href="http://www.rediris.es" target="_blank">http://www.rediris.es</a>)

\end{lstlisting}

\subsection{Día 26 12:41:14, Juan Manuel Macías}

\begin{lstlisting}
> Por cierto, la última entrada de tu bitácora: ¿es para una edición de
> una traducción al griego antiguo (o moderno, pero politónico) de «Don
> Quijote» con notas en español?
Lo de la traducción del Quijote es una historia un tanto novelesca. Se 
trata de una lectura anotada (prolijamente anotada) de una traducción 
anónima al griego moderno (politónico) contenida en un manuscrito de 
finales del XVIII / principios del XIX. La autora del trabajo descubrió 
este manuscrito de pura casualidad en la biblioteca Genadio de Atenas, 
olvidado y arrinconado, cuando andaba realizando otras investigaciones, 
así que se trata de una traducción inédita.
Un saludo,
Juan Manuel
El 26/02/17 a las 12:16, Pablo Rodríguez escribió:
> On 02/26/2017 11:02 AM, Juan Manuel Macías wrote:
>> Hola, Pablo. Muchas gracias por todo.
> De nada, Juan Manuel.
>
> ...(texto omitido)...
-- 
*Juan Manuel Macías*
http://diosas-nubes.blogspot.com <http://diosas-nubes.blogspot.com/>
https://apuntestipograficos.wordpress.com 
<https://apuntestipograficos.wordpress.com/>
http://www.revistacuadernoatico.com <http://www.revistacuadernoatico.com/>
Los artículos de ES-TEX son distribuidos gracias al apoyo y colaboración 
técnica de RedIRIS - Red Académica española - (http://www.rediris.es)

\end{lstlisting}
\section{Latex Beginners Guide}

\subsection{Día 24 19:34:53, Santiago Higuera}

\begin{lstlisting}
Hola:
Hoy, hasta dentro de cinco horas, Packt Publishing regala el libro 
'Latex Beginners Guide' en el siguiente enlace:
<a href="https://www.packtpub.com/packt/offers/free-learning" target="_blank">https://www.packtpub.com/packt/offers/free-learning</a>
Espero que lleguéis a tiempo los que no lo tengáis.
Un saludo
Santiago Higuera
Para darse de baja ES-TEX pincha y envia el siguiente url
mailto:[conectar para ver]

\end{lstlisting}

\subsection{Día 24 19:52:12, alberto alejandro moyano}

\begin{lstlisting}
+1
Saludos
El 24 de febrero de 2017, 15:34, Santiago Higuera <<a href="/cgi-bin/wa?LOGON=A3%3Dind1702%26L%3DES-TEX%26E%3Dquoted-printable%26P%3D528329%26B%3D--001a113d59744dcb0105494b3a3f%26T%3Dtext%252Fplain%3B%2520charset%3DUTF-8%26header%3D1" target="_parent" >[conectar para ver]</a>>
escribió:
> Hola:
>
> Hoy, hasta dentro de cinco horas, Packt Publishing regala el libro 'Latex
> Beginners Guide' en el siguiente enlace:
> ...(texto omitido)...
Para darse de baja ES-TEX pincha y envia el siguiente url
mailto:[conectar para ver]

\end{lstlisting}

\subsection{Día 24 22:32:43, Olmo}

\begin{lstlisting}
Hola, much�simas gracias por avisar, lo he pillado, quedan
algo m�s de 2 horas seg�n indica. Ya s�lo la tabla de contenidos
me encanta c�mo est� hecha.
Saludos,
Olmo.
Para darse de baja ES-TEX pincha y envia el siguiente url
mailto:[conectar para ver]

\end{lstlisting}

\subsection{Día 25 06:43:04, Aradenatorix Veckhôm Awecaelus}

\begin{lstlisting}
Interesante sitio.
No sé si ya tengo ese libro, creo que no, pero seria interesante si
los que lo alcanzaron pudieran compartirlo.
Saludos
Si tiene algun problema con la utilizacion de la lista.
Pongase en contacto con nosotros a traves de:
<a href="/cgi-bin/wa?LOGON=A2%3Dind1702%26L%3DES-TEX%26F%3D%26S%3D%26P%3D39809" target="_parent" >[conectar para ver]</a>

\end{lstlisting}

\subsection{Día 25 09:11:07, Santiago Higuera}

\begin{lstlisting}
Hola:
El libro está sujeto a copyright y, por tanto, no se puede poner enlace 
al mismo en abierto. Si alguien está particularmente interesado, le 
puedo hacer llegar, a través de correo privado, una copia de mi ejemplar 
para su uso personal.
En cualquier caso quiero decir, que pese a que tengo el libro hace 
tiempo y otro más avanzado del mismo autor y editorial, el único libro 
que utilizo habitualmente desde que salió y el que mejores servicios me 
presta es 'LaTeX con palabras clave', de J. Mulero y J.M. Sepulcre, 
publicado por la Universitat D'Alacant y que se puede ver en el enlace 
[1]. Lo compré al poco de salir y lo utilizo casi cada día. Es un gran 
libro que recomiendo a todos.
Saludos
Santiago Higuera
[1] <a href="http://publicaciones.ua.es/publica/ficha.aspx?fndCod=LI9788497174275" target="_blank">http://publicaciones.ua.es/publica/ficha.aspx?fndCod=LI9788497174275</a>
El 25/02/17 a las 06:43, Aradenatorix Veckhôm Awecaelus escribió:
> Interesante sitio.
>
> No sé si ya tengo ese libro, creo que no, pero seria interesante si
> los que lo alcanzaron pudieran compartirlo.
> ...(texto omitido)...
Si tiene algun problema con la utilizacion de la lista.
Pongase en contacto con nosotros a traves de:
<a href="/cgi-bin/wa?LOGON=A2%3Dind1702%26L%3DES-TEX%26F%3D%26S%3D%26P%3D40190" target="_parent" >[conectar para ver]</a>

\end{lstlisting}

\subsection{Día 25 09:37:49, Aradenatorix Veckhôm Awecaelus}

\begin{lstlisting}
Gracias Santiago, entiendo el punto. Respecto al libro de Mulero y
Sepulcre, supe de él por esta lista hace tiempo, pero no he sabido de
nadie que lo comercialice en México ni de forma de comprarlo sin que
me cueste un dineral. El que he usado, que aunque me parece muy bueno
ya requiere una segunda edición a 20 años de haberse publicado es el
de Gabriel Valiente Feruglio: Composición de textos científicos con
LaTeX de la UPC.
Saludos
Si tiene algun problema con la utilizacion de la lista.
Pongase en contacto con nosotros a traves de:
<a href="/cgi-bin/wa?LOGON=A2%3Dind1702%26L%3DES-TEX%26F%3D%26S%3D%26P%3D40990" target="_parent" >[conectar para ver]</a>

\end{lstlisting}
\section{mal alineamiento óptico de márgenes}

\subsection{Día 26 17:37:55, Pablo Rodríguez}

\begin{lstlisting}
Hola Juan Manuel (y resto de colisteros),
abro un nuevo hilo como ejemplo de lo que no se debe hacer. No mancho el
hilo anterior con esto.
Adobe con su Minion Pro (yo la tengo de Adobe Reader), consigue el
alineamiento de márgenes (<a href="http://www.ousia.tk/cpsp.pdf)" target="_blank">http://www.ousia.tk/cpsp.pdf)</a>.
Ahora, el problema es que está activado siempre:
<a href="http://www.ousia.tk/disable-cpsp.pdf)" target="_blank">http://www.ousia.tk/disable-cpsp.pdf)</a>. Da mal resultado.
Lo hacen con cpsp
(<a href="https://www.microsoft.com/typography/otspec/features_ae.htm#cpsp)" target="_blank">https://www.microsoft.com/typography/otspec/features_ae.htm#cpsp)</a>, que
es espaciado de mayúsculas. Me temo que está mal aplicado.
A ver si en la lista de OpenType me entero de qué intentaban hacer.
Respecto a tipografía, si los grandes hacen esto, ¿qué hará el resto?
Cuidaos de la malas tipografías ;-),
Pablo
-- 
<a href="http://www.ousia.tk" target="_blank">http://www.ousia.tk</a>
Los artículos de ES-TEX son distribuidos gracias al apoyo y colaboración 
técnica de RedIRIS - Red Académica española - (<a href="http://www.rediris.es" target="_blank">http://www.rediris.es</a>)

\end{lstlisting}

\subsection{Día 26 20:19:28, Juan Manuel Macías}

\begin{lstlisting}
Hola, Pablo,
Me acuerdo de haber visto esta anomalía en la Minion Pro hace tiempo, y 
no se me había ocurrido abrirla con Fontforge (la verdad es que es una 
fuente que nunca me gustó ;-). Parece que consigue el alineamiento por 
mero diseño del glifo, ¿no? Dejando los diacríticos fuera del margen 
izquierdo:
https://www.dropbox.com/sh/l2oewtg3m0ai2qq/AADE3zwrFU1nDyjsETzRAeava?dl=0
Pero claro, como dices, el problema es que el efecto de desplazamiento 
aparece siempre... ¿No había una fuente de la GFS que hacía algo 
parecido? Ahora no recuerdo, ¿La GFS Didot tal vez? Recuerdo que me lo 
habías comentado hace tiempo en mi bitácora...
He visto que en la Minion se aplica cpsp a mayúsculas griegas con y sin 
diacrítico, cuando en un texto con mayúsculas en griego sólo se usan las 
que no llevan diacrítico, vaya despropósito...
Saludos,
Juan Manuel
El 26/02/17 a las 17:37, Pablo Rodríguez escribió:
> Hola Juan Manuel (y resto de colisteros),
>
> abro un nuevo hilo como ejemplo de lo que no se debe hacer. No mancho el
> hilo anterior con esto.
> ...(texto omitido)...
-- 
*Juan Manuel Macías*
http://diosas-nubes.blogspot.com <http://diosas-nubes.blogspot.com/>
https://apuntestipograficos.wordpress.com 
<https://apuntestipograficos.wordpress.com/>
http://www.revistacuadernoatico.com <http://www.revistacuadernoatico.com/>
Los artículos de ES-TEX son distribuidos gracias al apoyo y colaboración 
técnica de RedIRIS - Red Académica española - (http://www.rediris.es)

\end{lstlisting}

\subsection{Día 26 21:01:10, Pablo Rodríguez}

\begin{lstlisting}
Hola Juan Manuel,
es verdad, había un tipo GFS, no Didot, sino otro que parecía más
manuscrito (tenía unas cuantas ligaduras griegas), que hacía eso. Pero
eso sencillamente era que el margen derecho estaba mal definido.
A mí Minion Pro no me apasiona, pero no creo que esté mal diseñada. Por
ejemplo, Garamond Premier Pro creo que es empalagosa. Por poner otro
ejemplo de la misma casa Adobe.
Gracias a tu imagen me doy cuenta que el margen izquierdo del glifo no
está bien puesto. De hecho, la propiedad "cpsp" incluso mejoraría la cosa.
En realidad, puede que los valores de cpsp no estén mal definidos. El
problema no es la propiedad OpenType, es saber poner márgenes a los
diferentes glifos.
Lo he preguntado en la lista OpenType y me dijeron que era un fallo.
Como no había abierto el glifo, lo que no sabía es que el fallo no
estaba en "cpsp". Lo comentaré ahora.
No sé si éste será otro ejemplo de que la composición tipográfica (text
typesetting) no tiene necesariamente que ver con el diseño de tipos
digitales (typeface design).
Muchas gracias por la aclaración y saludos,
Pablo
On 02/26/2017 08:19 PM, Juan Manuel Macías wrote:
> Hola, Pablo,
> 
> Me acuerdo de haber visto esta anomalía en la Minion Pro hace tiempo, y
> no se me había ocurrido abrirla con Fontforge (la verdad es que es una
> ...(texto omitido)...
-- 
<a href="http://www.ousia.tk" target="_blank">http://www.ousia.tk</a>
Los artículos de ES-TEX son distribuidos gracias al apoyo y colaboración 
técnica de RedIRIS - Red Académica española - (<a href="http://www.rediris.es" target="_blank">http://www.rediris.es</a>)

\end{lstlisting}

\subsection{Día 26 21:45:01, Juan Manuel Macías}

\begin{lstlisting}
> No sé si éste será otro ejemplo de que la composición tipográfica (text
> typesetting) no tiene necesariamente que ver con el diseño de tipos
> digitales (typeface design).
Completamente de acuerdo. No sé qué criterio ha seguido Slimbach para 
diseñar así los diacríticos de las mayúsculas, porque imagino que habrá 
estado revisando ejemplos de tipografía griega antes de crear las letras 
griegas de su Minion. Sería al menos lo esperable. Slimbach es un gran 
diseñador pero, sí, a menudo cae en ese empalago que dices (creo que no 
se puede definir mejor). A mí la que me da auténtica dentera es la Arno 
griega, buf, un desmelene. Esa fuente me recuerda a aquello que decía 
Morison, que mal camino para un diseñador de tipos si lo que busca es 
crear algo para que todos le digan "muy bonito". Morison, por cierto, 
como Zapf y alguno más, rara avis donde confluye el diseñador de fuentes 
y el compositor tipográfico...
Un saludo,
Juan Manuel
El 26/02/17 a las 21:01, Pablo Rodríguez escribió:
> Hola Juan Manuel,
>
> es verdad, había un tipo GFS, no Didot, sino otro que parecía más
> manuscrito (tenía unas cuantas ligaduras griegas), que hacía eso. Pero
> ...(texto omitido)...
-- 
*Juan Manuel Macías*
http://diosas-nubes.blogspot.com <http://diosas-nubes.blogspot.com/>
https://apuntestipograficos.wordpress.com 
<https://apuntestipograficos.wordpress.com/>
http://www.revistacuadernoatico.com <http://www.revistacuadernoatico.com/>
Los artículos de ES-TEX son distribuidos gracias al apoyo y colaboración 
técnica de RedIRIS - Red Académica española - (http://www.rediris.es)

\end{lstlisting}

\subsection{Día 26 22:12:43, Pablo Rodríguez}

\begin{lstlisting}
On 02/26/2017 09:45 PM, Juan Manuel Macías wrote:
>> No sé si éste será otro ejemplo de que la composición tipográfica (text
>> typesetting) no tiene necesariamente que ver con el diseño de tipos
>> digitales (typeface design).
> 
> ...(texto omitido)...
No sabía que Minion Pro era de Slimbach mismo (¡qué chapuza en lo que
comentamos!).
De Slimbach conozco su reputación, que seguro que está muy merecida.
Ahora, Garmond Premier Pro es como comerse un bol de cereales con leche
condensada. Una indigestión del quince...
Una prueba que puedes hacer es cambiar FreeSans (o lo que sea que uses
para «Don Quijote» en griego) a Garamond Premier Pro. Imposible. Quedará
bien en un folletillo, pero es impensable para un libro.
De hecho, tengo la sospecha de lo siguiente. Tenía que hacer una
Garamond nueva para que estuviese protegido legalmente por diseño. No
estoy seguro, pero de otro modo no lo entiendo.
> A mí la que me da auténtica dentera es la Arno
> griega, buf, un desmelene. Esa fuente me recuerda a aquello que decía
> Morison, que mal camino para un diseñador de tipos si lo que busca es
> crear algo para que todos le digan "muy bonito". Morison, por cierto,
> ...(texto omitido)...
No conozco Arno. Y por lo que me dices, ni pienso molestarme ninguna
imagen en buscar nada en internet.
Frutiger, Morrison, Zapf y otros cuantos dibujaban los tipos de letra y
los usaban en metal (vamos, se manchaban las manos).
Zapf cuenta que Palatino salió del modo más natural de escribir las
letras. Hizo caligrafía durante muchísimos años, con un plumín especial.
Y cuenta que después de tres años, se dió cuenta de que cogía el plumín
mal. Se corrigió y creó diseños intemporales.
En la inmediatez del presente, diríamos que Zapf después de tres años de
coger mal el plumín era un fracasado. Sólo así diseño tipos intemporales
como Palatino u Optima.
De hecho, Palatino Greek creo que es muy mala. Él tiene caligrafiados
los diseños de Palatino para letras griegas. No sé por qué no se usaron.
Los he visto en algún vídeo de YouTube y son mucho mejores que los
digitalizados. Incluso me juego algo a que no son suyos.
Es que al final, jugar con vectores en una pantalla y crear lápiz o
plumín en un papel no va a ser lo mismo.
Un saludo,
Pablo
-- 
<a href="http://www.ousia.tk" target="_blank">http://www.ousia.tk</a>
Los artículos de ES-TEX son distribuidos gracias al apoyo y colaboración 
técnica de RedIRIS - Red Académica española - (<a href="http://www.rediris.es" target="_blank">http://www.rediris.es</a>)

\end{lstlisting}

\subsection{Día 27 08:04:37, Juan Manuel Macías}

\begin{lstlisting}
¡Qué fascinante lo de Zapf! Y me alegro que comentes eso de las letras 
griegas de la Palatino Linotype, porque ya pensaba que era el único que 
tenía problemas con ellas. El griego de la PL me resulta incomodísimo 
para leer, aparte de que los diacríticos, sobre todo en las minúsculas, 
parecen patas de pulga: a veces se necesitaría un telescopio para saber 
si un espíritu es áspero o suave. Y eso, la verdad, está muy por debajo 
de lo que se esperaría de la limpieza en el diseño de Zapf. Para más 
inri, se ha convertido en casi un estándar en la filología clásica, 
supongo que porque viene de fábrica en los Windows. No he visto los 
dibujos originales de Zapf, pero comparto tu sospecha, probablemente hay 
una segunda mano (más torpe) en la "vectorialización" de esos glifos. 
Basta un simple ejemplo en este pantallazo que he sacado: 
https://www.dropbox.com/s/miqlywaxibvdc79/palatino_linotype.png?dl=0
No soy ningún experto, pero la alfa, a pesar de tener la misma altura x 
que la 'a', da una sensación a ojos vistas como de que es más grande 
pero de cuerpo más fino, aparte del raquitismo del acento griego frente 
al acento latino...
Un saludo,
Juan Manuel
El 26/02/17 a las 22:12, Pablo Rodríguez escribió:
> On 02/26/2017 09:45 PM, Juan Manuel Macías wrote:
>>> No sé si éste será otro ejemplo de que la composición tipográfica (text
>>> typesetting) no tiene necesariamente que ver con el diseño de tipos
>>> digitales (typeface design).
> ...(texto omitido)...
-- 
*Juan Manuel Macías*
http://diosas-nubes.blogspot.com <http://diosas-nubes.blogspot.com/>
https://apuntestipograficos.wordpress.com 
<https://apuntestipograficos.wordpress.com/>
http://www.revistacuadernoatico.com <http://www.revistacuadernoatico.com/>
Archivos de ES-TEX: http://listserv.rediris.es/archives/es-tex.html

\end{lstlisting}

\subsection{Día 27 23:03:20, Pablo Rodríguez}

\begin{lstlisting}
Lo del griego de Palatino Linotype clama al cielo. No es sólo que a ti
te resulte incómodo de leer, es que es incómodo de leer.
El trazo de los glifos latinos y los griegos no tiene que ver. Es como
si hubiesen cambiado el plumín por uno mucho más estrecho. Los glifos
griegos son monstruos gigantes, aunque esqueléticos.
No es un estándar en la filología clásica, es que la gente no tiene
mucho sentido tipográfico. ¿No se usaba Graeca? No era antológicamente
fea, es que realmente era ilegible.
Aún así en el año 2002 Itsmo la usa para su versión bilingüe de la
«Poética» de Aristóteles, en lo que constituye un crimen de lesa
cultura. En 1999, habían publicado una versión bilingúe del «Menón» con
una solución no muy acompasada, pero sin Graeca (a lo mejor copiaron la
imagen del texto griego de Budé). Sospecho que siguen usando Graeca para
la versión bilingüe del «Fedro» de 2010. Lo que es seguro es que
Dykinson la usó para su versión bilingüe de 2009 del «Fedro» también.
Vamos, eso es no tener ni el más mínimo sentido tipográfico. Con el
debido respeto.
Personalmente, cuando combino TeX Gyre Pagella (que me parece mucho
mejor que Palatino Linotype) con glifos griegos, uso GFS Didot. Supongo
que será muy mejorable (la cursiva es inclinada). Pero tiene cuerpo
suficiente para acompañar a TeX Gyre Pagella.
Sinceramente creo que ni vectorizaron los diseños originales de Zapf. No
pueden haber hecho tal chapuza de unos originales realmente buenos.
Recuerdo que con la Palatino de URW++ (URW Palladio L), hubo una versión
alemana FPL Neu (de ahí se toma TeX Gyre Pagella [si no recuerdo mal o
no estoy equivocado]). Hace como más de tres lustros, alguien me pasó
unos glifos griegos diseñados para esa tipografía. Luego los deshechó y
yo no sé dónde tengo ese archivo TrueType, pero lo que recuerdo no era malo.
URW++ (antes de que los comprasen) añadió glifos griegos (y cirílicos) a
su versión de Palatino (entre otros tipos de letras). Puedes
comprobarlas con:
  git clone <a href="http://git.ghostscript.com/urw-core35-fonts.git" target="_blank">http://git.ghostscript.com/urw-core35-fonts.git</a>
Necesitas tener instalado git, pero a lo mejor ya lo usas. Sinceramente
creo que el diseño de los glifos griegos es paupérrimo (en todos los que
he visto).
Por cierto, al parecer EB Garamond sí que usa con profusión lfbd y rtbd.
Se lo leí a Adam Twardoch en un comentario sobre el tema “optical
bounds” o márgenes ópticos. Lo que he visto en EB Garamond es que la
prominencia en la línea está marcada en cada uno de los glifos de esa
tipografía (la normal, al menos).
Saludos,
Pablo
PS: el tema de las tipografías griegas da para hacer una tesis (o unas
cuantas películas de terror).
On 02/27/2017 08:04 AM, Juan Manuel Macías wrote:
> ¡Qué fascinante lo de Zapf! Y me alegro que comentes eso de las letras
> griegas de la Palatino Linotype, porque ya pensaba que era el único que
> tenía problemas con ellas. El griego de la PL me resulta incomodísimo
> para leer, aparte de que los diacríticos, sobre todo en las minúsculas,
> ...(texto omitido)...
-- 
<a href="http://www.ousia.tk" target="_blank">http://www.ousia.tk</a>
Archivos de ES-TEX: <a href="http://listserv.rediris.es/archives/es-tex.html" target="_blank">http://listserv.rediris.es/archives/es-tex.html</a>

\end{lstlisting}

\subsection{Día 28 04:11:16, Juan Manuel Macías}

\begin{lstlisting}
Cierto, cierto, la Palatino Linotype griega es un despropósito
descomunal. Lo de que se había convertido en un estándar en clásicas lo
decía medio de coña, pero es que, en el fondo, ha acabado siendo así.
Hasta tal punto que las revistas (la mayoría de ellas) piden a los
autores que les manden los originales en Palatino L. Y es que a los de
clásicas en España hay que darles de comer aparte. Han pasado de usar la
Graeca (y mil más, a cada cual más espantosa, pero sobre todo la Graeca)
a usar la PL. Lo curioso es que antes de Unicode había unas cuantas
fuentes excelentes, las de codificación WinGreek, de lo mejorcito en la
época para el griego, y nunca tuvieron calado en España, y es que debe
de ser que aquí nos gusta lo gore. De las WinGreek recuerdo algunas con
especial cariño, y siempre tengo pendiente ponerme a recodificarlas en
Unicode. Empecé un verano con la Grecs du Roi, pero luego me dio pereza
y no seguí. Tiempo después descubrí que Haralambous ya lo había hecho: 
http://www.polytoniko.org/GrecsDuRoiUnicode.ttf
Yo siempre intento recomendar a amigos o conocidos del mundillo de
clásicas que, si quieren usar una fuente versátil y legible para su
trabajo diario, dejen de lado la PL y que tiren por la Gentium, o al
menos por las Noto de Google, que también son legibles y prácticas.
Lo que me comentas de la edición del 99 de Menón en Istmo, no podría
poner ahora la mano en el fuego, y tal vez me equivoque, pero casi
seguro que fue como dices, reproduciendo la imagen del texto griego. No
era poco común hacer eso. Tengo ahora mismo aquí una edición de Taurus
(1987), bilingüe, de la Poética de Aristóteles y la de Horacio (Epístola
a los Pisones), y el texto griego es descaradamente una reproducción
facsímil de la edición de Oxford.
Lo de la Graeca, totalmente, daría para abrir un museo de los horrores.
Y aún sigue haciendo de las suyas por ahí. Y seguirá, porque bicho malo
nunca muere. Y es que es una fuente tan mala (casi hasta en el sentido
ético) que uno ya piensa si no estará diseñada adrede para hacer daño.
En cuanto a las fuentes griegas de estilo didot / aplá, la que más
tiendo a usar es la Old Standard de Kryukov. Con sus defectillos, me
parece una fuente muy equilibrada. No sé si el autor la ha dibujado de
cero o sobre plantilla (me inclino por lo segundo), pero me resulta un
estilo didot muy grato, parecido al que se usaba en España en los años
30, 40, 50 del pasado siglo (para representar griego). La encuentras,
por ejemplo, en los viejos tomos de la Espasa, en las primeras ediciones
de Alma Mater (no los abortos que producen ahora) o incluso en los
cuadernillos de ejercicios AΘΗΝΑ de Berenguer. Algo modificada por mí,
la he usado en algunos libros en combinación con la New Calededonia.
Aquí un ejemplo:
https://www.dropbox.com/s/lxmoqk5qeug0hif/Gromero.jpg?dl=0 y otro más:
https://www.dropbox.com/s/aewnf7j81fb1vu8/Bernabe.pdf?dl=0
Aunque en otras ocasiones he usado la Porson de la GFS, también con New
Caledonia: https://www.dropbox.com/s/l8non8q82wt277v/Gruskova.jpg?dl=0
Sí, tengo Git. Le echaré un vistazo al enlace que me mandas, mil
gracias. Lo de las películas de terror y la tipografía griega, qué gran
verdad. El exorcista al lado sería Sonrisas y lágrimas ;-)
Saludos,
Juan Manuel
El 27/02/17 a las 23:03, Pablo Rodríguez escribió:
> Lo del griego de Palatino Linotype clama al cielo. No es sólo que a ti
> te resulte incómodo de leer, es que es incómodo de leer.
>
> El trazo de los glifos latinos y los griegos no tiene que ver. Es como
> ...(texto omitido)...
-- 
*Juan Manuel Macías*
http://diosas-nubes.blogspot.com <http://diosas-nubes.blogspot.com/>
https://apuntestipograficos.wordpress.com
<https://apuntestipograficos.wordpress.com/>
http://www.revistacuadernoatico.com <http://www.revistacuadernoatico.com/>

\end{lstlisting}

\subsection{Día 28 21:46:44, Pablo Rodríguez}

\begin{lstlisting}
Grecs du Roi está bien, pero yo no me puedo plantear usarla para un
libro. ¿Leerías la Ilíada con ese tipo de letra?
Mi problema con Palatino Linotype es la legibilidad pura y simple. Igual
que con Gentium, no me acaba de convencer. Y eso que la he usado. Tiene
un diseño excesivamente marcado.
Otra combinación que uso es la GaramondNo8 de URW con Theano Didot. Para
mí, Old Standard no tiene su elegancia en los glifos griegos. Siento ser
tan categórico, pero creo que es infinitamente más legible. La cursiva
de Old Standard (que es real, no mera inclinación) la uso para mezclarla
con Theano Didot.
El daño de Graeca es la poca idea que tiene el personal en general.
Siento ser tan categórico. Ahora si en "α Ἦ α Η" se muestra un error
básico en Minion Pro, no sé qué le vamos a pedir a la gente. No creo que
el mismo Slimbach haya metido ese gazapo, lo fastidiado es que lo ha
firmado.
A ver si en unos días puedo poner un ejemplo (además, de ediciones
críticas).
Saludos,
Pablo
On 02/28/2017 04:11 AM, Juan Manuel Macías wrote:
> Cierto, cierto, la Palatino Linotype griega es un despropósito
> descomunal. Lo de que se había convertido en un estándar en clásicas lo
> decía medio de coña, pero es que, en el fondo, ha acabado siendo así.
> Hasta tal punto que las revistas (la mayoría de ellas) piden a los
> ...(texto omitido)...
-- 
<a href="http://www.ousia.tk" target="_blank">http://www.ousia.tk</a>

\end{lstlisting}

\subsection{Día 28 22:35:33, Juan Manuel Macías}

\begin{lstlisting}
> Grecs du Roi está bien, pero yo no me puedo plantear usarla para un
> libro. ¿Leerías la Ilíada con ese tipo de letra?
Todavía no he encontrado el libro que pudiera llevar la Grecs du Roi. 
Probablemente exista, y espero descubrirlo algún día. Pero en una 
edición crítica sería un suicidio. O en un estudio filológico. Pero no 
es una fuente para nada ilegible, bien al contrario. Y eso que estaba 
inspirada en las cursivas bizantinas que podían llegar a ser 
absolutamente lisérgicas. Por cierto, la letra de Yannis Ristsos era muy 
parecida a la cursiva bizantina. Hace siglos se leían sin problema las 
ediciones de Robert Estienne con la Grecs du Roi, que la verían como lo 
más natural, sin los énfasis que el tiempo ha dejado en ella; y la 
original era todo una orgía de ligaduras. El revival de Mindaugas 
Strockis en los 90 es una pequeña sombra de todo ese alarde. Son unos 
tipos muy anacrónicos. A la Grecs du Roi le pasa lo mismo que a las 
fraktur (hoy día, ojo, teniendo en cuenta que las fraktur se usaron 
hasta casi antes de ayer). Pero sí, probablemente habrá un libro que 
pueda componerse con Grecs du Roi. ¿Cuál? Es cuestión de descubrirlo.
> Otra combinación que uso es la GaramondNo8 de URW con Theano Didot. Para
> mí, Old Standard no tiene su elegancia en los glifos griegos. Siento ser
> tan categórico, pero creo que es infinitamente más legible. La cursiva
> de Old Standard (que es real, no mera inclinación) la uso para mezclarla
> ...(texto omitido)...
La Theano Didot es una buena fuente, pero la veo en muchos aspectos como 
a "medio terminar". Siento disentir en esto, pues creo que la Old 
Standard es muchísimo mejor. Por versatilidad, por ser más completa. Por 
acabado. Es una fuente de batalla, que es lo que han sido siempre las 
fuentes aplá y te aseguro que en el trabajo real saca las castañas del 
fuego como ninguna. Ahora mismo (salvo que venga algo mejor) es mi 
opción número uno en este estilo. Coincido, desde luego, en que la 
cursiva es una maravilla. Ahora bien, en cuanto a lo de combinar una 
Garamond con una didona, yo diría que no es una buena práctica. Son 
diseños antagónicos. Las Garamond son de trazos más delicados que no 
pegan ni con cola con los contrastes tan definidos de didonas y bodonis. 
Con fuentes como la Palatino, y otros especímenes modernos, pueden 
aceptarse los experimentos y hasta cierta heterodoxia. Y aún así la 
Palatino sigue pegándose de palos con el estilo aplá. Pero con fuentes 
con un acento histórico tan marcado, conviene seguir la vieja norma (que 
creo que tiene bastante sentido común): didonas y bodonis con didonas y 
bodonis, o transicionales (tal vez alguna Baskerville o alguna Century), 
pero nunca didonas y bodonis con garaldas o venecianas. Pero es sólo mi 
opinión. Por otra parte, y también es una opinión muy personal mía, si 
por mí fuera haría desaparecer las Garamond de la faz de la tierra al 
menos por una centuria, o dejar sólo la Sabon, que no es una Garamond, 
sino una "traducción" moderna de la Garamond.
Saludos,
Juan Manuel
El 28/02/17 a las 21:46, Pablo Rodríguez escribió:
> Grecs du Roi está bien, pero yo no me puedo plantear usarla para un
> libro. ¿Leerías la Ilíada con ese tipo de letra?
>
> Mi problema con Palatino Linotype es la legibilidad pura y simple. Igual
> ...(texto omitido)...
-- 
*Juan Manuel Macías*
http://diosas-nubes.blogspot.com <http://diosas-nubes.blogspot.com/>
https://apuntestipograficos.wordpress.com 
<https://apuntestipograficos.wordpress.com/>
http://www.revistacuadernoatico.com <http://www.revistacuadernoatico.com/>

\end{lstlisting}

\end{document}
